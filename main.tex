%%
% The SUEPThesis Template for Bachelor Graduation Thesis
%
% 上海电力大学毕业设计(论文)中英文摘要 —— 使用 XeLaTeX 编译
%
% Copyright 2020-2023 SUEPaper
%
% This work may be distributed and/or modified under the
% conditions of the LaTeX Project Public License, either version 1.3
% of this license or (at your option) any later version.
% The latest version of this license is in
%   http://www.latex-project.org/lppl.txt
% and version 1.3 or later is part of all distributions of LaTeX
% version 2005/12/01 or later.
%
% This work has the LPPL maintenance status `maintained'.
%
% The Current Maintainer of this work is Haiwen Zhang.
%%

\documentclass{suepthesis}

\SUEPSetup{
  cover = {
    %% 使用以下参数来自定义封面日期
    date = 2023年6月5日,
  },
  info = {
    title = 水下机器人生产和存储计划,
    % 当且仅当封面题目的第一栏写不下,再使用subtitle用于扩充
    subtitle = 问题的优化建模,
    titleEn = {Optimal modeling of production and storage planning for WPCR},
    institution = 数理学院数学系,
    major = 信息与计算科学专业2019级,
    author = 徐逸宁,
    studentId = 20192295,
    supervisor = 邓化宇,
    keywords = {动态规划;0-1规划;层次分析;遗传算法;策略迭代},
    keywordsEn = {Dynamic Programming; 0-1 Programming; Analytic Hierarchy Process; Genetic Algorithm; Strategy Iteration},
  }
}

% 使用 BibLaTeX 处理参考文献
%   biblatex-gb7714-2015 常用选项
%     gbnamefmt=lowercase     姓名大小写由输入信息确定
%     gbpub=false             禁用出版信息缺失处理
\usepackage[
  backend=biber,
  style=gb7714-2015,
  gbnamefmt=lowercase,
  gbpub=false
]{biblatex}     

% 参考文献引用文件位于 content/thesis.bib
\addbibresource{content/thesis.bib}

% 脚注格式
\usepackage[perpage,bottom,hang]{footmisc}

% 定义图片文件目录与扩展名
\graphicspath{{figures/}{images/}}
\DeclareGraphicsExtensions{.pdf,.eps,.png,.jpg,.jpeg}

% 使用三线表:toprule,midrule,bottomrule。
\usepackage{booktabs}

% 表格中支持跨行
\usepackage{multirow}

% 表格中数字按小数点对齐
\usepackage{dcolumn}
\newcolumntype{d}[1]{D{.}{.}{#1}}

% 使用长表格
\usepackage{longtable}

% 附带脚注的表格
\usepackage{threeparttable}

% 附带脚注的长表格
\usepackage{threeparttablex}

% 算法环境宏包
\usepackage[ruled,vlined,linesnumbered]{algorithm2e}

% 直立体数学符号
\providecommand{\dd}{\mathop{}\!\mathrm{d}}
\providecommand{\ee}{\mathrm{e}}
\providecommand{\ii}{\mathrm{i}}
\providecommand{\jj}{\mathrm{j}}

% 国际单位制宏包
\usepackage{siunitx}[=v2]

% 绘图宏包
\usepackage{tikz}
\usetikzlibrary{positioning, shapes.geometric}


% 文档开始
\begin{document}

% 标题页面:如无特殊需要,本部分无需改动
\MakeCover

% 原创性声明:如无特殊需要,本部分无需改动
% ====== 原创性声明(LaTeX 格式)======
\MakeOriginality
% ====== 原创性声明(LaTeX 格式)======

% 前置页面定义
\frontmatter
% 摘要:在摘要相应的 TeX 文件处进行摘要部分的撰写
%%
% The SUEPThesis Template for Bachelor Graduation Thesis
%
% 上海电力大学毕业设计(论文)中英文摘要 —— 使用 XeLaTeX 编译
%
% Copyright 2020-2023 SUEPaper
%
% This work may be distributed and/or modified under the
% conditions of the LaTeX Project Public License, either version 1.3
% of this license or (at your option) any later version.
% The latest version of this license is in
%   http://www.latex-project.org/lppl.txt
% and version 1.3 or later is part of all distributions of LaTeX
% version 2005/12/01 or later.
%
% This work has the LPPL maintenance status `maintained'.
%
% The Current Maintainer of this work is Haiwen Zhang.
%%

% 中英文摘要章节
\begin{abstract}

  本文研究的是基于生产与存储优化问题的WPCR的组装计划问题。

  首先,在无存货无结余的情况安排组装生产,且无需组装等待,
  将每次生产准备费用和单件库存费用及每天WPCR需求和关键设备工时限制引入约束条件,
  采用0-1变量规划分段函数,以层次分析法进行权重判断,
  将多约束的多任务动态规划问题简化为在不同权重下的最优求解问题,
  并将中间结果输出显示,最终得到最小费用为6260.9元。

  其次,加入了需要生产等待的前提条件,重新规划时间序列,
  首先在不允许缺货情形WPCR生产和存储模型求解结果的基础上同样进行0-1规划和分段函数的设定,
  然后采用顺序递推的方法进行循环判断,为防止陷入局部最优解的情况,
  对设置不同初值的情况进行对比分析,最终得到全局最优解,经过多次求解得最小费用为207658元。

  然后,增加检修时间节点,同时生产周期大幅延长,故而在模型上增设衰减系数以及检修约定的0-1规划,
  对连续多周生产统筹规划的WPCR生产和存储模型进行补充,而后依据启发式算法中的遗传算法进行迭代优化,
  通过修改迭代次数以及种群适应性调节,随机设置初始时间节点,通过多次解算对比得出全局最优解,
  选取第24天、第72天、第99天、第120天、第153天、第171天和第202天,最小费用为6308701元。

  最后,将连续30周的WPCR需求数据作为历史数据在需求未知的情况下进行需求和生产分配,
  选取带有反馈调节过程的策略迭代算法进行模型的自适应动态规划的训练,同时使用遗传算法进行训练加强,
  通过修改种群变异率、种群规模参数并多次求解对比,
  在连续多周生产统筹规划的WPCR生产和存储模型的基础上得出全局最优解,其最小费用为194879元。
  
\end{abstract}
  
% 英文摘要章节
\begin{abstractEn}

  In this paper, the assembly planning problem of WPCR based on production and storage optimization is studied.

  Firstly, the assembly and production are arranged in the case of no inventory and no balance,
  and there is no need to wait for assembly, the data of each production preparation cost, 
  one-piece inventory cost, 
  daily WPCR requirement and key equipment man-hour limit are introduced into the constraint conditions, 
  the 0-1 variable programming segmentation function is adopted, 
  and the weight judgment is carried out by the analytic hierarchy method, 
  and the multi-task dynamic programming problem with multiple constraints is simplified to the optimal 
  solution problem under different weights, and the intermediate result output is displayed, 
  and the final minimum cost is 6260.9 yuan.

  Secondly, the precondition of needing production waiting is added, the time series is replanned,
  first 0-1 planning and segmentation function setting are also carried out on the basis of WPCR 
  production and storage model solution results that are not allowed to be out of stock, 
  and then the sequential recursion method is used for circular judgment, in order to prevent falling 
  into the situation of local optimal solution, the situation of setting different initial values is 
  compared and analyzed, and finally the global optimal solution is obtained, and the minimum cost is 
  207658 yuan after multiple solutions.

  Then, the maintenance time node is increased, and the production cycle is greatly extended, 
  so the attenuation coefficient and the 0-1 planning of the maintenance agreement are added to the 
  model to supplement the WPCR production and storage model of continuous multi-week production 
  overall planning, and then iterative optimization is carried out according to the genetic algorithm 
  in the heuristic algorithm, the initial time node is randomly set by modifying the number of iterations
  and population adaptability adjustment, and the global optimal solution is obtained by multiple solution 
  comparisons, and the 24th day, day 72, 99th day, day 120, day 24, day 72, day 99, day 120, On days 153,
  171 and 202, the minimum fee is 6308701 yuan.

  Finally, WPCR demand data for 30 consecutive weeks is used as historical data for demand and production
  allocation under the condition of unknown demand, and the strategy iteration algorithm with feedback
  adjustment process is selected to train the adaptive dynamic programming of the model, and at the same 
  time the genetic algorithm is used for training and strengthening, and the global optimal solution is 
  obtained on the basis of the WPCR production and storage model of continuous multi-week production overall
  planning by modifying the population variation rate and population size parameters and solving and 
  comparing them for multiple times, and the minimum cost is 194879 yuan.

\end{abstractEn}
  

\MakeTOC

% 正文开始
\mainmatter

%%
% The SUEPThesis Template for Bachelor Graduation Thesis
%
% 上海电力大学毕业设计(论文)中英文摘要 —— 使用 XeLaTeX 编译
%
% Copyright 2020-2023 SUEPaper
%
% This work may be distributed and/or modified under the
% conditions of the LaTeX Project Public License, either version 1.3
% of this license or (at your option) any later version.
% The latest version of this license is in
%   http://www.latex-project.org/lppl.txt
% and version 1.3 or later is part of all distributions of LaTeX
% version 2005/12/01 or later.
%
% This work has the LPPL maintenance status `maintained'.
%
% The Current Maintainer of this work is Haiwen Zhang.
%%

%子章节为了便于查找和修改,建议通过input拆分文件

%%
% The BIThesis Template for Bachelor Graduation Thesis
%
% 上海电力大学毕业设计(论文)中英文摘要 —— 使用 XeLaTeX 编译
%
% Copyright 2020-2023 SUEPaper
%
% This work may be distributed and/or modified under the
% conditions of the LaTeX Project Public License, either version 1.3
% of this license or (at your option) any later version.
% The latest version of this license is in
%   http://www.latex-project.org/lppl.txt
% and version 1.3 or later is part of all distributions of LaTeX
% version 2005/12/01 or later.
%
% This work has the LPPL maintenance status `maintained'.
%
% The Current Maintainer of this work is Haiwen Zhang.
%%

\chapter{一级标题}

这是上海电力大学本科学位论文\LaTeX{}模板,下面的文字主要作用为对重构后的模板样式设置进行测试。
测试样例基本覆盖模板设定,包括多级标题的基本样式,段落与缩进距离。

\section{二级标题}

\subsection{三级标题}

\subsubsection{四级标题}

一级标题根据学校提供的Word模板要求,三号黑体居中,上下各空一行,章节号空一个汉字,
并且每一章节单独起一页,章节号格式应使用阿拉伯数字而非中文汉字。

二级标题为小四号黑体,缩进两个汉字。章节号后空一个汉字。

三级标题小四号楷体GB2312,字体包含在项目中,同样缩进两个汉字,章节号后空一个汉字。

四级标题参照本科学术论文设计样式,分项采取(1)、(2)、(3)的序号。

所有标题样式由 suepthesis.cls 模板文件 ctexset 进行设置。

\section{字体}

正文字体默认使用小四号宋体,英文为小四号 Times New Romen,各段行首缩进两个汉字



英文字体展示如下:

TeX (/tɛx, tɛk/, see below), stylized within the system as TEX, is a typesetting system (or a "formatting system") which was designed and mostly written by Donald Knuth\cite{knuth1984texbook} and released in 1978. TeX is a popular means of typesetting complex mathematical formulae; it has been noted as one of the most sophisticated digital typographical systems.


\subsection{调节字号}

可以使用 命令来调节字号。

\begin{tabular}{ll}
  \verb|\zihao{3} | & \zihao{3}  三号字 English \\
  \verb|\zihao{-3}| & \zihao{-3} 小三号 English \\
  \verb|\zihao{4} | & \zihao{4}  四号字 English \\
  \verb|\zihao{-4}| & \zihao{-4} 小四号 English \\
  \verb|\zihao{5} | & \zihao{5}  五号字 English \\
  \verb|\zihao{-5}| & \zihao{-5} 小五号 English \\
\end{tabular}

\subsection{调节字体}

需要说明的是由于学校写作指导要求的字体部分不可在Linux上使用,即便你的写作过程是在Linux或者macOS上完成的,
我们仍\textbf{强烈建议}您在Windows操作系统上编译最终版论文。

中文可选字体以及选用指令如下:

\begin{tabular}{l l}
  \verb|\songti| & {\songti 宋体} \\
  \verb|\heiti| & {\heiti 黑体}  \\
  \verb|\kaiti| & {\kaiti 楷体}
\end{tabular}

我们在模板中通过调整\verb|\newCJKfontfamily|的AutoFakeBold参数来简单实现字体加粗,在正文中你可以使用习惯的\verb|\textbf|指令来加粗对应中文。如果你需要调整字体,也可以组合使用字体选择并加粗,比如\verb|\kaiti\bfseries|

\textbf{宋体加粗测试},宋体不加粗测试。

{\kaiti\bfseries 楷体加粗测试。}{\kaiti 楷体不加粗测试。}

目前模板并没有按照一些其他模板写法中常见的,重定向加粗和倾斜效果到upright,TODO:后续可能考虑重定义\verb|emph|和\verb|strong|样式。




\section{模板主要结构}

本项目模板的主要结构, 如下表所示:
% TODO 进一步完善

\begin{table}[ht]
  \centering
  \begin{tabular}{r|l|l}
    \hline\hline
    \multicolumn{2}{l|}{main.tex } & 主文档,可以理解为文章入口。  \\ \hline
                                            & abstact.tex    & 中/英文摘要内容    \\ \cline{2-3}
                                            & acknowledgements.tex       & 致谢 \\ \cline{2-3}
                                            & appendix.tex       & 附录 \\ \cline{2-3}
                                            & conclusion.tex       & 结论 \\ \cline{2-3}
                                            & ref.bib       & 参考文献库 \\ \cline{2-3}
                                            & reference.tex       & 参考文献 \\ \cline{2-3}
    \raisebox{1em}{content 目录 }            & chapters 目录   & 章节内容           \\ \hline
    \multicolumn{2}{l|}{images 目录}         & 用于存放图片文件                                  \\ \hline
    \multicolumn{2}{l|}{suepthesis.cls }    & 模板入口                                          \\ \hline\hline
  \end{tabular}
\end{table}

我们不建议模板使用者更改原有模板的结构,
但如果您确实需要,请务必先充分阅读本模板的使用说明并了解相应的\LaTeX{}模板设计知识。

%%
% The SUEPThesis Template for Bachelor Graduation Thesis
%
% 上海电力大学毕业设计(论文)中英文摘要 —— 使用 XeLaTeX 编译
%
% Copyright 2020-2023 SUEPaper
%
% This work may be distributed and/or modified under the
% conditions of the LaTeX Project Public License, either version 1.3
% of this license or (at your option) any later version.
% The latest version of this license is in
%   http://www.latex-project.org/lppl.txt
% and version 1.3 or later is part of all distributions of LaTeX
% version 2005/12/01 or later.
%
% This work has the LPPL maintenance status `maintained'.
%
% The Current Maintainer of this work is Haiwen Zhang.
%%

\chapter{不允许缺货的WPCR生产和存储模型}

\section{不允许缺货的WPCR生产和存储模型的建立}

下面研究$n$阶段生产存储计划,$d_k$设为第$k$阶段对产品WPCR的需求,根据题目设置如下约定:
\begin{equation}
    C_j(x_k)=k_j + c_j \cdot x_k
\end{equation}
\begin{equation}
    T_k=\sum_{j=1}t_j \cdot x_k
\end{equation}
\begin{equation}
    t_k \leq T_k
\end{equation}

$x_k$表示第$k$天的产量,$c_j$表示第$j$种单间库存费用,$k_j$表示第$j$种生产准备费用,$C_j$表示第$j$种成本费用,
$t_j$表示第$j$种单件消耗工时,$t_k$表示第$k$天生产所有组件所耗时间,
$T_k$表示第$k$天生产所有组件总工时限制。

\subsection{生产和库存平衡关系}

设为$x_k$第$k$阶段WPCR的生产量,$v_k$为第$k$阶段结束时WPCR的库存量,则对于WPCR而言,有
\begin{equation}
    v_k = v_{k-1} + x_k - d_k
\end{equation}

再结合该工厂第一天(周一)开始时没有任何组件库存,也不希望第7天(周日)结束后留下任何组件库存,则
\begin{equation}
    \begin{cases}
        v_0=0 \\ 
        v_k=v_{k-1} + x_k - d_k,k=1,2,...,n \\ 
        v_n=0
    \end{cases}
\end{equation}

某工厂生产的WPCR装置需要用3个容器艇(用A表示)、4个机器臂(用B表示)、5个动力系统(用C表示)组装而成。

对A而言,第$k$阶段生产量为$x_k^A$,第$k$阶段结束时A的库存量为$v_k^A$,消耗量为$x_k^A=3x_k$,
则根据A的生产和消耗平衡关系,有
\begin{equation}
    v_k^A=v_{k-1}^A + x_k^A - 3x_k
\end{equation}

对B而言,第$k$阶段生产量为$x_k^B$,第$k$阶段结束时B的库存量为$v_k^B$,消耗量为$x_k^B=4x_k$,
则根据B的生产和消耗平衡关系,有
\begin{equation}
    v_k^B=v_{k-1}^B + x_k^B - 4x_k
\end{equation}

对C而言,第k阶段生产量为$x_k^C$,第k阶段结束时C的库存量为$v_k^C$,消耗量为$x_k^C=5x_k$,则根据C的生产和消耗平衡关系,有
\begin{equation}
    v_k^C=v_{k-1}^C + x_k^C - 5x_k
\end{equation}

同理,对于A1、A2、A3、B1、B2、C1、C2、C3可以类似研究,设生产一件WPCR所需要的A1、A2、A3、B1、B2、C1、C2、C3分别为
$u_{A1}$,$u_{A2}$,$u_{A3}$,$u_{B1}$,$u_{B2}$,$u_{C1}$,$u_{C2}$,$u_{C3}$,则
\begin{equation}
    v_k^{Ai}=v_{k-1}^{Ai} + x_k^{Ai} - u_{Ai}x_k,i=1,2,3
\end{equation}
\begin{equation}
    v_k^{Bi}=v_{k-1}^{Bi} + x_k^{Bi} - u_{Bi}x_k,i=1,2
\end{equation}
\begin{equation}
    v_k^{Ci}=v_{k-1}^{Ci} + x_k^{Ci} - u_{Ci}x_k,i=1,2
\end{equation}

具体地,有
\begin{equation}
    \begin{cases}
        v_k^{A1}=v_{k-1}^{A1} + x_k^{A1} - 18x_k \\ 
        v_k^{A2}=v_{k-1}^{A2} + x_k^{A2} - 24x_k \\ 
        v_k^{A3}=v_{k-1}^{A3} + x_k^{A3} - 6x_k \\ 
        v_k^{B1}=v_{k-1}^{B1} + x_k^{B1} - 8x_k \\ 
        v_k^{B2}=v_{k-1}^{B2} + x_k^{B2} - 16x_k \\ 
        v_k^{C1}=v_{k-1}^{C1} + x_k^{C1} - 40x_k \\ 
        v_k^{C2}=v_{k-1}^{C2} + x_k^{C2} - 10x_k \\ 
        v_k^{C3}=v_{k-1}^{C3} + x_k^{C3} - 60x_k
    \end{cases}
\end{equation}

\subsection{不允许缺货约束}

按照工厂的信誉要求,目前接收的所有订单到期必须全部交货,轻易不能有缺货事件发生。
引入不允许缺货模型,或者设缺货损失为无穷大,则有如下条件限制。
\begin{equation}
    s.t.
    \begin{cases}
        v_k \geqslant 0, V_k^A \geqslant 0,V_k^B \geqslant 0,V_k^C \geqslant 0, & k=1,2,...,n \\
        v_k^{A1} \geqslant 0, V_k^{A2} \geqslant 0,V_k^{A3} \geqslant 0, & k=1,2,...,n \\
        v_k^{B1} \geqslant 0, V_k^{B2} \geqslant 0, & k=1,2,...,n \\
        v_k^{C1} \geqslant 0, V_k^{C2} \geqslant 0,V_k^{C3} \geqslant 0, & k=1,2,...,n \\
    \end{cases}
\end{equation}

\subsection{总工时约束}

已知A、B、C的工时消耗分别为3时/件、5时/件和5时/件,即生产1件A需要占用3个工时,
生产1件B需要占用5个工时,生产1件C需要占用5个工时。

第$k$阶段$x_k^A$、$x_k^A$、$x_k^A$的生产量为分别为,所需总工时不能超过总工时$T_k$的限制,
即工时约束条件为
\begin{equation}
    3x_k^A + 5x_k^B + 5x_k^C \leq T_k
\end{equation}

\subsection{WPCR各组件的生产准备成本和存储成本}

为了顺利生产WPCR,工厂在某一天生产组件产品时,需要付出一个与生产数量无关的固定成本,称为生产准备费用。
比如第一天生产了A,则要支付A的生产准备费用,若第二天再生产A,则需要再支付A的生产准备费用。

首先根据表\ref{T.ch2-1}可知,单件WPCR生产准备费用为240元,单件WPCR库存费用为5元。
生产组件A的生产准备费用为120元,库存费用为2元,组件B的生产准备费用为160元,
库存费用为1.5元,组件C的生产准备费用为180元,库存费用为1.7元。
生产零件A1的生产准备费用为40元,库存费用为5元,生产零件A2的生产准备费用为60元,
库存费用为3元,生产零件A3的生产准备费用为50元,库存费用为6元。
生产零件B1的生产准备费用为80元,库存费用为4元,生产零件B2的生产准备费用为100元,库存费用为5元。
生产零件C1的生产准备费用为60元,库存费用为3元,生产零件C2的生产准备费用为40元,库存费用为2元,
生产零件C3的生产准备费用为70元,库存费用为3元。

\begin{table}[!hpt]
    \caption{每次生产准备费用和单件库存费用(单位:元)}
    \label{T.ch2-1}
    \centering
    \renewcommand\arraystretch{1.5} 
    \begin{tabular}{@{}ccccccccccccc@{}} 
    \toprule
    \textbf{产品} & \textbf{WPCR} & \textbf{A} & \textbf{A1} & \textbf{A2} & \textbf{A3} & \textbf{B} & \textbf{B1} & \textbf{B2} & \textbf{C} & \textbf{C1} & \textbf{C2} & \textbf{C3} \\
    \midrule
    \textbf{生产准备费用} & 240 & 120 & 40 & 60 & 50 & 160 & 80 & 100 & 180 & 60 & 40 & 70 \\ 
    \textbf{单件库存费用} & 5 & 2 & 5 & 3 & 6 & 1.5 & 4 & 5 & 1.7 & 3 & 2 & 3 \\
    \bottomrule
    \end{tabular}
\end{table}

设$c_k(x_k)$表示第$k$阶段生产$x_k$的WPCR时的生产准备费用,则
\begin{equation}
    c_k(x_k)=\begin{cases}
        0, & x_k=0\\
        240, & x_k>0
    \end{cases}
\end{equation}

引入$0-1$变量$y_k$和任意大正数$M$(取相对$10^3$倍的数)
\begin{equation}
    y_k=\begin{cases}
        1, & x_k>0,\text{表示生产}x_k\text{生产准备费用不为}0 \\
        0, & x_k=0,\text{表示不生产}x_k\text{生产准备费用为}0
    \end{cases}
\end{equation}
且满足
\begin{equation}
    x_k \leq My_k
\end{equation}

则生产准备费用表达式等价于
\begin{equation}
    c_k(x_k)=240y_k
\end{equation}
注意到$v_k$为第$k$阶段结束时WPCR的库存量,单件WPCR的库存费用为5元/件,则库存费用为
\begin{equation}
    h_k(v_k)=5v_k
\end{equation}

这样对于WPCR的费用为
\begin{equation}
    C_{\uppercase\expandafter{\romannumeral1}}=\sum_{k=1}^{n}[C_k(x_k)+h_k(v_k)]=\sum_{k=1}^{n}(240y_k+5v_k)
\end{equation}

同理研究A、B、C的生产准备费和库存费用。引入$0-1$变量$y_k^A$,$y_k^B$,$y_k^C$,即
\begin{equation}
    y_k^A=\begin{cases}
        1, & x_k^A>0\\
        0, & x_k^A=0
    \end{cases}
\end{equation}
\begin{equation}
    y_k^B=\begin{cases}
        1, & x_k^B>0\\
        0, & x_k^B=0
    \end{cases}
\end{equation}
\begin{equation}
    y_k^C=\begin{cases}
        1, & x_k^C>0\\
        0, & x_k^C=0
    \end{cases}
\end{equation}
且满足
\begin{equation}
    x_k^A \leq My_k^A,x_k^B \leq My_k^B,x_k^C \leq My_k^C
\end{equation}
则对于A、B、C的生产和存储费用为
\begin{equation}
    C_{\uppercase\expandafter{\romannumeral2}}=\sum_{k=1}^{n}(120y_k^A+2v_k^A+160y_k^B+1.5v_k^B+180y_k^C+1.7v_k^C)
\end{equation}

再进一步研究A1、A2、A3、B1、B2、C1、C2、C3的生产准备费和库存费用。
引入$0-1$变量$y_k^{A1}$,$y_k^{A2}$,$y_k^{A3}$,$y_k^{B1}$,$y_k^{B2}$,$y_k^{C1}$,$y_k^{C2}$,$y_k^{C3}$即
\begin{equation}
    y_k^{Ai}=\begin{cases}
        1, & x_k^{Ai}>0\\
        0, & x_k^{Ai}=0
    \end{cases},i=1,2,3
\end{equation}
\begin{equation}
    y_k^{Bi}=\begin{cases}
        1, & x_k^{Bi}>0\\
        0, & x_k^{Bi}=0
    \end{cases},i=1,2
\end{equation}
\begin{equation}
    y_k^{Ci}=\begin{cases}
        1, & x_k^{Ci}>0\\
        0, & x_k^{Ci}=0
    \end{cases},i=1,2,3
\end{equation}
且满足
\begin{equation}
    \begin{cases}
        x_k^{Ai} \leq My_k^{Ai}, &i=1,2,3 \\
        x_k^{Bi} \leq My_k^{Bi},&i=1,2 \\
        x_k^{Ci} \leq My_k^{Ci},&i=1,2,3
    \end{cases}
\end{equation}
则对于A1、A2、A3、B1、B2、C1、C2、C3的生产和存储费用为
\begin{equation}
    \begin{aligned}
        C_{\uppercase\expandafter{\romannumeral3}} &= 
            \sum_{k=1}^{n}(40y_k^{A1}+5v_k^{A1}+60y_k^{A2}+3v_k^{A2}+50y_k^{A3}+6v_k^{A3}) \\
            &+ \sum_{k=1}^{n}(80y_k^{B1}+4v_k^{B1}+100y_k^{B2}+5v_k^{B2}) \\
            &+ \sum_{k=1}^{n}(60y_k^{C1}+3v_k^{C1}+40y_k^{C2}+2v_k^{C2}+70y_k^{C3}+3v_k^{C3})
    \end{aligned}
\end{equation}



\subsection{不允许缺货的WPCR生产和存储模型}

汇总1-20,1-25,1-30三式得目标函数为
\begin{equation}
    \text{min}\sum_{k=1}^{n}(C_{\uppercase\expandafter{\romannumeral1}}
        + C_{\uppercase\expandafter{\romannumeral2}} 
        + C_{\uppercase\expandafter{\romannumeral3}})
\end{equation}
综上,所有限制条件总结如下:
\begin{equation}
    s.t.
    \begin{cases}
        v_k=v_{k-1}+x_k-d_k,v_0=0 \\
        3x_k^A+5x_k^B+5x_k^C \leq T_k \\
        ... \\
        y_k,y_k^A,y_k^B,y_k^C,y_k^{Ai},y_k^{Bi},y_k^{Ci}=0\text{或}1
    \end{cases} \qquad k=1,2,...,n
\end{equation}

\section{不允许缺货情形WPCR生产和存储模型的求解}
表2-2为每天WPCR需求和关键设备工时限制,可以看出周一到周日WPCR的需求分别为39、36、38、40、37、33、40,
同时A、B、C周一到周日生产总工时限制分别为4500、2500、2750、2100、2500、2750、1500。
通过编写Lingo程序求解结果为表2-3所示。

从表2-2可以看出,周一没有库存费用,只有生产准备费用。周二只有关于B组件的组装计划,而周三只有组件B不需要组装。
周四和周日没有生产计划,只有库存费用。周五和周六的生产计划比较全面,生产准备费用和库存费用都需要。
同时,有该表可知,如果生产计划很全面,WPCR、A、B、C都需要组装,无论其中需要组装的数量是多少,
生产准备费用都不变,都为1200元。总的生产和存储成本为6260.9元。
%%
% The SUEPThesis Template for Bachelor Graduation Thesis
%
% 上海电力大学毕业设计(论文)中英文摘要 —— 使用 XeLaTeX 编译
%
% Copyright 2020-2023 SUEPaper
%
% This work may be distributed and/or modified under the
% conditions of the LaTeX Project Public License, either version 1.3
% of this license or (at your option) any later version.
% The latest version of this license is in
%   http://www.latex-project.org/lppl.txt
% and version 1.3 or later is part of all distributions of LaTeX
% version 2005/12/01 or later.
%
% This work has the LPPL maintenance status `maintained'.
%
% The Current Maintainer of this work is Haiwen Zhang.
%%

\chapter{连续多周生产统筹规划的WPCR生产和存储模型}

\section{连续多周生产统筹规划模型的建立}

令$V_{i-1}$为状态变量,表示第$i$天开始时的库存量;$x_i$为决策变量,表示第$i$天的生产量。
状态转移方程为
\begin{equation*}
    v_i=v_{i-1}+x_i-d_i \qquad i=1,2,...,n
\end{equation*}

最优值函数$f_i(v_i)$表示从第一天初始库存量为$0$到第$i$天末库存量为$v_i$时的最小总费用,
因而可写出顺序递推关系式\cite{韩中庚2007实用运筹学模型}:
\begin{equation}
    f_i(v_i)=\mathop{\min}_{0\leq x_i \leq \sigma_i}[C_i(x_i)+h_i(v_i)+f_{i-1}(V_{i-1})] \qquad i=1,2,...,n
\end{equation}

其中,
\begin{equation}
    \sigma_i=\min(v_i+d_i,m)
\end{equation}

不允许缺货的WPCR生产和存储模型在无存货无结余的情况下进行一周(7天)的机器人的组装计划,
求解总成本最小。

然而,事实上,组件A、B、C需要提前一天生产入库才能组装WPCR,
A1、A2、A3、B1、B2、C1、C2、C3也需要提前一天生产入库才能组装A、B、C。
在连续多周生产情况下,需要统筹规划。
由不允许缺货的WPCR生产和存储模型的求解结果列出库存量和生产准备成本如下:
\begin{equation}
    v_k=v_{k-1}+x_k-d_k
\end{equation}
\begin{equation}
    C_k(x_k)=\begin{cases}
        0, & x_k=0 \\
        K+c_j \cdot x_k, & x_k=1,2,...,n \\
        \infty, & x_k > m
    \end{cases}
\end{equation}

$d_k$表示第$k$阶段对组件的需求量,$v_k$表示第$k$阶段结束时组件库存量,$m$表示每个阶段最多能生产该组件的上限数。
\begin{equation}
    \min g = \sum_{n}^{k=1}[C_k(x_k)+h_k(v_k)]
\end{equation}
\begin{equation}
    \begin{cases}
        v_0=0,v_n=0 \\
        v_k=\sum_{k}^{i}(x_i - d_i) \ge 0 & k=2,3,...,n-1 \\
        0 \leq x_k \leq m & k=1,2,...,n \\
        x_k \text{为整数} & k=1,2,...,n
    \end{cases}
\end{equation}

$h_k(v_k)$表示第$k$阶段结束时有库存量$v_k$所需的存储费用。
\begin{equation}
    y_k=\begin{cases}
        1, & x_k > 0, \text{表示生产}x_k,\text{生产准备费不为}0 \\
        0, & x_k = 0, \text{表示不生产}x_k,\text{生产准备用费为}0 
    \end{cases}
\end{equation}
\begin{equation}
    C=\sum_{n}^{k=1}[C_k(x_k)+h_k{v_k}]=\sum_{k=1}^{n}[K \cdot y_k + c_j \cdot x_k]
\end{equation}

对于WPCR而言,
\begin{equation}
    v_k=v_{k-1} + x_k - d_k    
\end{equation}

对于A,B,C组件而言,根据A,B,C生产与组装关系,有
\begin{equation}
    v_{k+2}^A=v_{k+1}^A + x_{k+2}^A - 3x_{k+2}
\end{equation}
\begin{equation}
    v_{k+2}^B=v_{k+1}^B + x_{k+2}^B - 4x_{k+2}
\end{equation}
\begin{equation}
    v_{k+2}^C=v_{k+1}^C + x_{k+2}^C - 5x_{k+2}
\end{equation}

同理,对于A1、A2、A3、B1、B2、C 1、C2、C3可以类似研究,设生产一件WPCR所需要的
A1、A2、A3、B1、B2、C1、C2、C3分别为$u_{A1}$,$u_{A2}$,$u_{A3}$,$u_{B1}$,
$u_{B2}$,$u_{C1}$,$u_{C2}$,$u_{C3}$,则
\begin{equation}
    v_{k+1}^{Ai}=v_k^{Ai} + x_{k+1}^{Ai} - u_{Ai}3x_k,\quad i=1,2,3
\end{equation}
\begin{equation}
    v_{k+1}^{Bi}=v_k^{Bi} + x_{k+1}^{Bi} - u_{Bi}3x_k,\quad i=1,2
\end{equation}
\begin{equation}
    v_{k+1}^{Ci}=v_k^{Ci} + x_{k+1}^{Ci} - u_{Ci}3x_k,\quad i=1,2,3
\end{equation}

$s.t.$
\begin{equation}
    3x_{k+2}^A+5x_{k+2}^B+5x_{k+2}^C \leq T_k
\end{equation}
\begin{equation}
    y_k=\begin{cases}
        1, & x_k > 0, \text{表示生产}x_k,\text{生产准备费不为}0 \\
        0, & x_k = 0, \text{表示不生产}x_k,\text{生产准备用费为}0 
    \end{cases}
\end{equation}
\begin{equation}
    x_k \leq My_k
\end{equation}

则各费用为,
\begin{equation}
    C_{\uppercase\expandafter{\romannumeral1}}^{'}=
        \sum_{n}^{k=1}[C_k(x_k)+h_k{v_k}]=
        \sum_{n}^{k=1}(240y_k+5v_k)
\end{equation}
\begin{equation}
    C_{\uppercase\expandafter{\romannumeral2}}^{'}=
        \sum_{n}^{k=1}(120y_{k+2}^A+2v_{k+2}^A+160y_{k+2}^B+1.5v_{k+2}^B+180y_{k+2}^C+1.7v_{k+2}^C)
\end{equation}

\begin{equation}
    \begin{aligned}
        C_{\uppercase\expandafter{\romannumeral3}}^{'} &= 
            \sum_{k=1}^{n}(40y_{k+1}^{A1}+5v_{k+1}^{A1}+60y_{k+1}^{A2}+3v_{k+1}^{A2}+50y_{k+1}^{A3}+6v_{k+1}^{A3}) \\
            &+ \sum_{k=1}^{n}(80y_{k+1}^{B1}+4v_{k+1}^{B1}+100y_{k+1}^{B2}+5v_{k+1}^{B2}) \\
            &+ \sum_{k=1}^{n}(60y_{k+1}^{C1}+3v_{k+1}^{C1}+40y_{k+1}^{C2}+2v_{k+1}^{C2}+70y_{k+1}^{C3}+3v_{k+1}^{C3})
    \end{aligned}
\end{equation}



汇总三式得目标函数为
\begin{equation}
    \min\sum_{k=1}^{n}(
            C_{\uppercase\expandafter{\romannumeral1}}^{'}+
            C_{\uppercase\expandafter{\romannumeral2}}^{'}+
            C_{\uppercase\expandafter{\romannumeral3}}^{'}) 
\end{equation}

\section{连续多周生产统筹规划模型的求解}

基于以下表\ref{T.ch3-1}的连续30周的WPCR需求的数据,
带入到连续多周生产统筹规划模型中并求解,得出表\ref{T.ch3-2}的求解结果表格。

\begin{longtable}[c]{c*{7}{r}}
    \caption{考虑检修的生产和存储模型求解结果展示}
    \label{T.ch3-1} \\
    \toprule
    \textbf{天} & \textbf{周一} & \textbf{周二} & \textbf{周三} & 
    \textbf{周四} & \textbf{周五} & \textbf{周六} & \textbf{周日} \\
    \midrule
    \endfirsthead
    \multicolumn{8}{l}{\textbf{续表~\thetable}} \\

    \toprule
    \textbf{天} & \textbf{周一} & \textbf{周二} & \textbf{周三} & 
    \textbf{周四} & \textbf{周五} & \textbf{周六} & \textbf{周日} \\
    \midrule
    \endhead
    \hline
    \multicolumn{8}{r}{续下页}
    \endfoot
    \endlastfoot
    \textbf{第1周}  & 1 & 2 & 3 & 4 & 5 & 6 & 7 \\ 
    \textbf{第2周}  & 1 & 2 & 3 & 4 & 5 & 6 & 7 \\ 
    \textbf{第3周}  & 1 & 2 & 3 & 4 & 5 & 6 & 7 \\ 
    \textbf{第4周}  & 1 & 2 & 3 & 4 & 5 & 6 & 7 \\
    \textbf{第5周}  & 1 & 2 & 3 & 4 & 5 & 6 & 7 \\
    \textbf{第6周}  & 1 & 2 & 3 & 4 & 5 & 6 & 7 \\
    \textbf{第7周}  & 1 & 2 & 3 & 4 & 5 & 6 & 7 \\
    \textbf{第8周}  & 1 & 2 & 3 & 4 & 5 & 6 & 7 \\ 
    \textbf{第9周}  & 1 & 2 & 3 & 4 & 5 & 6 & 7 \\ 
    \textbf{第10周}  & 1 & 2 & 3 & 4 & 5 & 6 & 7 \\
    \textbf{第11周}  & 1 & 2 & 3 & 4 & 5 & 6 & 7 \\
    \textbf{第12周}  & 1 & 2 & 3 & 4 & 5 & 6 & 7 \\
    \textbf{第13周}  & 1 & 2 & 3 & 4 & 5 & 6 & 7 \\
    \textbf{第14周}  & 1 & 2 & 3 & 4 & 5 & 6 & 7 \\
    \textbf{第15周}  & 1 & 2 & 3 & 4 & 5 & 6 & 7 \\
    \textbf{第16周}  & 1 & 2 & 3 & 4 & 5 & 6 & 7 \\
    \textbf{第17周}  & 1 & 2 & 3 & 4 & 5 & 6 & 7 \\
    \textbf{第18周}  & 1 & 2 & 3 & 4 & 5 & 6 & 7 \\
    \textbf{第19周}  & 1 & 2 & 3 & 4 & 5 & 6 & 7 \\
    \textbf{第20周}  & 1 & 2 & 3 & 4 & 5 & 6 & 7 \\
    \textbf{第21周}  & 1 & 2 & 3 & 4 & 5 & 6 & 7 \\
    \textbf{第22周}  & 1 & 2 & 3 & 4 & 5 & 6 & 7 \\
    \textbf{第23周}  & 1 & 2 & 3 & 4 & 5 & 6 & 7 \\
    \textbf{第24周}  & 1 & 2 & 3 & 4 & 5 & 6 & 7 \\
    \textbf{第25周}  & 1 & 2 & 3 & 4 & 5 & 6 & 7 \\
    \textbf{第26周}  & 1 & 2 & 3 & 4 & 5 & 6 & 7 \\
    \textbf{第27周}  & 1 & 2 & 3 & 4 & 5 & 6 & 7 \\
    \textbf{第28周}  & 1 & 2 & 3 & 4 & 5 & 6 & 7 \\
    \textbf{第29周}  & 1 & 2 & 3 & 4 & 5 & 6 & 7 \\ 
    \textbf{第30周}  & 1 & 2 & 3 & 4 & 5 & 6 & 7 \\ 
    \bottomrule
\end{longtable}

\begin{table}[!hpt]
    \caption{连续多周生产统筹规划的WPCR生产和存储模型求解的结果}
    \label{T.ch3-2}
    \centering
    \renewcommand\arraystretch{1.5} 
    \begin{tabular}{@{}ccccccc@{}} 
    \toprule
    \textbf{日期} & \multicolumn{1}{c}{\textbf{WPCR}} & \multicolumn{1}{c}{\textbf{A组装}} 
    & \multicolumn{1}{c}{\textbf{B组装}} & \multicolumn{1}{c}{\textbf{C组装}}
    & \multicolumn{1}{c}{\textbf{生产准备}} & \multicolumn{1}{c}{\textbf{库存}} \\
                & \multicolumn{1}{c}{\textbf{组装数量}} & \multicolumn{1}{c}{\textbf{数量}} 
    & \multicolumn{1}{c}{\textbf{数量}} & \multicolumn{1}{c}{\textbf{数量}}
    & \multicolumn{1}{c}{\textbf{费用}} & \multicolumn{1}{c}{\textbf{费用}} \\
    \midrule
    \textbf{周一} & 237 & 316 & 395 & 41 & 700 & 1619.5 \\
    \textbf{周二} & 237 & 316 & 395 & 41 & 700 & 1619.5 \\
    \textbf{周三} & 237 & 316 & 395 & 41 & 700 & 1619.5 \\
    \textbf{周四} & 237 & 316 & 395 & 41 & 700 & 1619.5 \\
    \textbf{周五} & 237 & 316 & 395 & 41 & 700 & 1619.5 \\
    \textbf{周六} & 237 & 316 & 395 & 41 & 700 & 1619.5 \\
    \textbf{周日} & 237 & 316 & 395 & 41 & 700 & 1619.5 \\
    \textbf{总和} & 237 & 316 & 395 & 41 & 700 & 1619.5 \\ 
    \bottomrule
    \end{tabular}
\end{table}



从表\ref{T.ch3-2}可以看出,连续多周生产统筹规划模型中周一、周二、周五、周六、周日的生产计划很全面,
和不允许缺货的WPCR生产和存储模型一样,如果生产计划很全面,WPCR、A、B、C都需要组装,
无论其中需要组装的数量是多少,生产准备费用都不变,都为1200元。周三只进行WPCR组装的生产计划,
而周四进行组件A、B、C的组装生产计划。总的生产和存储成本为207658元,
连续多周生产统筹规划模型平均单周所用的成本为6921.93元,
要高于不允许缺货的WPCR生产和存储模型每周的生产成本。
%%
% The BIThesis Template for Bachelor Graduation Thesis
%
% 上海电力大学毕业设计(论文)中英文摘要 —— 使用 XeLaTeX 编译
%
% Copyright 2020-2023 SUEPaper
%
% This work may be distributed and/or modified under the
% conditions of the LaTeX Project Public License, either version 1.3
% of this license or (at your option) any later version.
% The latest version of this license is in
%   http://www.latex-project.org/lppl.txt
% and version 1.3 or later is part of all distributions of LaTeX
% version 2005/12/01 or later.
%
% This work has the LPPL maintenance status `maintained'.
%
% The Current Maintainer of this work is Haiwen Zhang.
%%

\chapter{公式与符号}

\LaTeX 的公式环境中符号样式符合 \TeX 默认的美国数学学会(AMS)的符号使用习惯,中文论文写作推荐遵循 GB/T 3102.11——1993《物理科学和技术中的数学符号》标准。这里我们给出一些 \LaTeX 中常用的符号表示。


\section{\LaTeX 数学公式模式}

\LaTeX 提供了两种数学公示的写作模式:内联模式和独显模式:

\begin{itemize}
    \item \textbf{内联模式}(inline mode),又称为行内模式,随文模式,将公式显示为段落的一部分。
    \item \textbf{独显模式}(display mode),又称为行间模式,将公式用独立行展示出来,不再作为段落的一部分。
\end{itemize}

\subsection{内联模式}

% TODO

键入如下定义符之一在段落中来使用内联模式书写数学公式符号:

\begin{itemize}
    \item \verb|\(...\)|
    \item \verb|$...$|
    \item \verb|\begin{math}...\end{math}|
\end{itemize}

\subsection{独显模式}

使用如下方式以独显模式表示数学公式:

\begin{itemize}
    \item \verb|\[...\]|
    \item \verb|\begin{displaymath}...\end{displaymath}|
    \item \verb|\begin{equation}...\end{equation}|
\end{itemize}

\textbf{公式插入示例如公式(\ref{E.example})所示。}

\begin{equation}
\gamma_{x}=
\left\{
  \begin{array}{lr}
  0, & {\rm if}~~\;|x| \leq \delta \\
  x, & {\rm otherwise}
  \end{array}
\right.
\label{E.example}
\end{equation}


\newpage


%%
% The BIThesis Template for Bachelor Graduation Thesis
%
% 上海电力大学毕业设计(论文)中英文摘要 —— 使用 XeLaTeX 编译
%
% Copyright 2020-2023 SUEPaper
%
% This work may be distributed and/or modified under the
% conditions of the LaTeX Project Public License, either version 1.3
% of this license or (at your option) any later version.
% The latest version of this license is in
%   http://www.latex-project.org/lppl.txt
% and version 1.3 or later is part of all distributions of LaTeX
% version 2005/12/01 or later.
%
% This work has the LPPL maintenance status `maintained'.
%
% The Current Maintainer of this work is Haiwen Zhang.
%%

\chapter{代码示例}

\section{minted包}

为了实现代码高亮,使用了LaTeX的minted宏包,minted宏包使用Python的第三方库Pygments,
相对于listing宏包Pygments提供了更好的代码语法高亮功能,因此需要在你的电脑里安装Pygments库。

\section{Python代码示例}

\inputminted[
    frame=lines,
    framesep=2mm,
    baselinestretch=1.2,
    fontsize=\small,
    linenos
]{python}{code/demo.py}

\section{C++代码示例}
\inputminted[
    frame=lines,
    framesep=2mm,
    baselinestretch=1.2,
    fontsize=\small,
    linenos
]{cpp}{code/demo.cpp}
%%
% The SUEPThesis Template for Bachelor Graduation Thesis
%
% 上海电力大学毕业设计(论文)中英文摘要 —— 使用 XeLaTeX 编译
%
% Copyright 2020-2023 SUEPaper
%
% This work may be distributed and/or modified under the
% conditions of the LaTeX Project Public License, either version 1.3
% of this license or (at your option) any later version.
% The latest version of this license is in
%   http://www.latex-project.org/lppl.txt
% and version 1.3 or later is part of all distributions of LaTeX
% version 2005/12/01 or later.
%
% This work has the LPPL maintenance status `maintained'.
%
% The Current Maintainer of this work is Haiwen Zhang.
%%

\chapter{不同情形WPCR生产和存储模型的综合分析}

\section{不同情形的WPCR生产和存储总成本比较}

通过前面四个模型优化结果可知,连续多周生产统筹规划模型相较于不允许缺货的WPCR生产和存储模型增加组装迟滞,
由图表可以看出其费用大大升高,对研究组装分配计划具有重要现实意义。
考虑检修的生产和存储模型和WPCR不确定生产和存储模型在连续多周生产统筹规划模型的基础上展开,
分别得出增设检修日期和需求未知条件下的最小费用,经对比分析,可对复杂生产中组装计划的调度具有指导意义。

\begin{figure}[h]
    \centering
    \includegraphics[width=0.7\linewidth]{ch6-1.png}
    \caption{各问题优化结果比较}
    \label{f.ch6-1}
\end{figure}

\section{不同情形的WPCR生产和存储求解的灵敏度分析}

经过灵敏度分析验证,通过遗传算法的迭代次数,以及种群数、变异率、交叉率以求得最短运行时间下的最优结果,
得出模型运算灵敏度分析,模型在大于150次迭代运算的情况下运行结果基本一致,在策略迭代中采取大种群,
在费用估计中可用最优变异交叉率来节省运算时间来求得全局最优解和动态稳定解。

\begin{figure}[H]
    \centering
    \begin{subfigure}[t]{0.5\linewidth}
        \captionsetup{justification=centering}
        \begin{minipage}[b]{1\linewidth}
            \centering
            \includegraphics[width=0.8\linewidth]{ch6-2-1.png}
            \caption{}
        \end{minipage}
    \end{subfigure}
    \hspace{-5em}
    \begin{subfigure}[t]{0.5\linewidth}
        \captionsetup{justification=centering}
        \begin{minipage}[b]{1\linewidth}
            \centering
            \includegraphics[width=0.8\linewidth]{ch6-2-2.png}
            \caption{}
        \end{minipage}
    \end{subfigure}\\
    \begin{subfigure}[t]{0.5\linewidth}
        \captionsetup{justification=centering}
        \begin{minipage}[b]{1\linewidth}
            \centering
            \includegraphics[width=0.8\linewidth]{ch6-2-3.png}
            \caption{}
        \end{minipage}
    \end{subfigure}
    \hspace{-5em}
    \begin{subfigure}[t]{0.5\linewidth}
        \captionsetup{justification=centering}
        \begin{minipage}[b]{1\linewidth}
            \centering
            \includegraphics[width=0.8\linewidth]{ch6-2-3.png}
            \caption{}
        \end{minipage}
    \end{subfigure}
    \caption{不同迭代次数对比}
    \label{f.ch6-2}
\end{figure}
%%
% The SUEPThesis Template for Bachelor Graduation Thesis
%
% 上海电力大学毕业设计(论文)中英文摘要 —— 使用 XeLaTeX 编译
%
% Copyright 2020-2023 SUEPaper
%
% This work may be distributed and/or modified under the
% conditions of the LaTeX Project Public License, either version 1.3
% of this license or (at your option) any later version.
% The latest version of this license is in
%   http://www.latex-project.org/lppl.txt
% and version 1.3 or later is part of all distributions of LaTeX
% version 2005/12/01 or later.
%
% This work has the LPPL maintenance status `maintained'.
%
% The Current Maintainer of this work is Haiwen Zhang.
%%

\chapter{结论}

\section{模型评价}

(1)模型构建了需求与生产的反馈过程及在需求随机的情况下,优化生产组装计划,进一步拓展了模型的适用范围。

(2)考虑到总工时的动态改变,设计了一种带末端优化的遗传算法求解,并构建了一种双层规划模型,通过修改部分参数,
模型能同时适用于多种场景。

(3)模型的自适应能力强,针对未来订单未知情形的WPCR不确定生产和存储模型加入了自适应动态规划,
在遗传训练下具有更高的适应性和推广性,对供求关系动态平衡的研究具有一定的数学理论价值。

\section{模型的缺点}

(1)只考虑了生产总工时,没有考虑有多少生产机器以及如何安排生产顺序,在实际应用中考虑不够全面。

(2)总费用只考虑了生产准备费及存储费用,没有考虑人力、物力、时间等成本。

(3)本文A、B、C消耗工时固定,未考虑组件加工过程中有可能出现的时间差。

\section{模型的改进}

(1)研究在更多约束条件下,完成多任务、多目标的混合优化模型,增加模型的适应性和广谱性,
进而探讨实际生产中供求平衡和成本优化问题。

(2)发展模型的逆向运用,即已知生产成本的前提下,解决具体生产分配问题。

% 后置部分
\backmatter

% 结论:在结论相应的 TeX 文件处进行结论部分的撰写
% \input{content/conclusion.tex}
% 致谢:在致谢相应的 TeX 文件处进行致谢部分的撰写
%%
% The BIThesis Template for Bachelor Graduation Thesis
%
% 上海电力大学毕业设计(论文)中英文摘要 —— 使用 XeLaTeX 编译
%
% Copyright 2020-2023 SUEPaper
%
% This work may be distributed and/or modified under the
% conditions of the LaTeX Project Public License, either version 1.3
% of this license or (at your option) any later version.
% The latest version of this license is in
%   http://www.latex-project.org/lppl.txt
% and version 1.3 or later is part of all distributions of LaTeX
% version 2005/12/01 or later.
%
% This work has the LPPL maintenance status `maintained'.
%
% The Current Maintainer of this work is Haiwen Zhang.
%%

\begin{acknowledgements} 

感谢最先制作出中南大学博士学位论文 LaTeX 模板的郭大侠@CSGrandeur。

感谢添加本科学位论文样式支持的@BlurryLight。

感谢帮助重构项目并进行测试的@burst-bao以及为独立使用LaTeX进行毕业论文写作提供宝贵经验的16级的姜析阅学长。

感谢所有为模板贡献过代码的同学们!

\end{acknowledgements}

% 参考文献:如无特殊需要,参考文献相应的 TeX 文件无需改动,添加参考文献请使用 BibTeX 的格式
%   添加至 misc/ref.bib 中,并在正文的相应位置使用 \cite{xxx} 的格式引用参考文献
\input{content/reference.tex}
% 附录:在附录相应的 TeX 文件处进行附录部分的撰写
%%
% The SUEPThesis Template for Bachelor Graduation Thesis
%
% 上海电力大学毕业设计(论文)中英文摘要 —— 使用 XeLaTeX 编译
%
% Copyright 2020-2023 SUEPaper
%
% This work may be distributed and/or modified under the
% conditions of the LaTeX Project Public License, either version 1.3
% of this license or (at your option) any later version.
% The latest version of this license is in
%   http://www.latex-project.org/lppl.txt
% and version 1.3 or later is part of all distributions of LaTeX
% version 2005/12/01 or later.
%
% This work has the LPPL maintenance status `maintained'.
%
% The Current Maintainer of this work is Haiwen Zhang.
%%

\begin{appendices}
    
\appendixsection{不允许缺货的WPCR生产和存储模型求解的Lingo源程序}
\inputminted[
    frame=lines,
    framesep=2mm,
    baselinestretch=1.2,
    fontsize=\small
]{matlab}{code/demo.m}


\newpage
\appendixsection{符号说明}

\begin{center}
    \begin{longtable}{m{8cm}m{7cm}}
        \toprule
        \textbf{符号}&\textbf{说明}\\
        \midrule
        \endfirsthead
        \toprule
        \textbf{符号}&\textbf{说明}\\
        \midrule
        \endhead 
        \bottomrule
        \endfoot
        \bottomrule
        \endlastfoot

        $d_k$ & 第$k$阶段对组件的需求量 \\
        $x_k$ & 第$k$天的产量 \\ 
        $c_j$ & 第$j$种单间库存费用 \\ 
        $k_j$ & 第$j$种生产准备费用 \\ 
	\end{longtable}
\end{center}
  
\end{appendices}


\end{document}
