\chapter{SUEPThesis简介}

\textbf{SUEPThesis}由上海电力大学数学系纸上得来终觉浅团队开发。
本文是\textbf{SUEPThesis}的示例文档,基本上覆盖了模板中所有格式的设置。
建议大家在使用模板之前,仔细阅读本文档,以及示例论文。

\section{二级标题}

\subsection{三级标题}

一级标题根据学校提供的Word模板要求,小三号黑体居中,章节号空一个汉字,1.5 倍行距,段前0行,段后0.5行
并且每一章节单独起一页,章节号格式应使用中文汉字。

二级标题为黑体四号字,居左,1.5倍行距,段前0.5行,段后0行。标题号后空一个汉字。

三级标题黑体小四号字,1.5倍行距,段前0.5行,段后0行,缩进两个汉字,标题号后空一个汉字。

所有标题样式由 \textbf{suepthesis.cls} 模板文件 \textbf{ctexset} 进行设置。

\section{字体}

正文字体默认使用小四号宋体,英文为小四号 Times New Romen,各段行首缩进两个汉字。

英文字体展示如下:

TeX (/tɛx, tɛk/, see below), stylized within the system as TEX, 
is a typesetting system (or a "formatting system") which was designed and mostly written by
Donald Knuth\cite{knuth1984texbook} and released in 1978. 
TeX is a popular means of typesetting complex mathematical formulae; 
it has been noted as one of the most sophisticated digital typographical systems.


\subsection{调节字号}

可以使用 命令来调节字号。

\begin{tabular}{ll}
  \verb|\chuhao | & \chuhao  初号字 English \\
  \verb|\xiaochu| & \xiaochu 小初号 English \\
  \verb|\yihao  | & \yihao  一号字 English \\
  \verb|\xiaoyi | & \xiaoyi 小一号 English \\
  \verb|\erhao  | & \erhao  二号字 English \\
  \verb|\xiaoer | & \xiaoer 小二号 English \\
  \verb|\sanhao | & \sanhao  三号字 English \\
  \verb|\xiaosan| & \xiaosan 小三号 English \\
  \verb|\sihao  | & \sihao  四号字 English \\
  \verb|\xiaosi | & \xiaosi 小四号 English \\
  \verb|\wuhao  | & \wuhao  五号字 English \\
  \verb|\xiaowu | & \xiaowu 小五号 English \\
  \verb|\liuhao  | & \liuhao  六号字 English \\
  \verb|\xiaoliu | & \xiaoliu 小六号 English \\
  \verb|\qihao  | & \qihao  七号字 English \\
  \verb|\bahao | & \bahao 八号字 English \\
\end{tabular}

\subsection{调节字体}

需要说明的是由于Linux或者macOS系统上不包含学校写作指导要求的字体,因此我们将Windows的字体放在了fonts目录下,
由于这些字体具有版权,所以我们仍\textbf{强烈建议}您在Windows操作系统上编译最终版论文。

中文可选字体以及选用指令如下:

\begin{tabular}{l l}
  \verb|\songti| & {\songti 宋体} \\
  \verb|\heiti| & {\heiti 黑体}
\end{tabular}

我们在模板中通过调整\verb|\newCJKfontfamily|的AutoFakeBold参数来简单实现字体加粗,
在正文中你可以使用习惯的\verb|\textbf|指令来加粗对应中文。
如果你需要调整字体,也可以组合使用字体选择并加粗,比如\verb|\heiti\bfseries|

\textbf{宋体加粗测试},宋体不加粗测试。

{\heiti\bfseries 黑体加粗测试。}{\heiti 黑体不加粗测试。}


\textcolor{blue}{\zhlipsum 阅后删除蓝色文字。}
