%%
% The BIThesis Template for Bachelor Graduation Thesis
%
% 上海电力大学毕业设计(论文)中英文摘要 —— 使用 XeLaTeX 编译
%
% Copyright 2020-2023 SUEPaper
%
% This work may be distributed and/or modified under the
% conditions of the LaTeX Project Public License, either version 1.3
% of this license or (at your option) any later version.
% The latest version of this license is in
%   http://www.latex-project.org/lppl.txt
% and version 1.3 or later is part of all distributions of LaTeX
% version 2005/12/01 or later.
%
% This work has the LPPL maintenance status `maintained'.
%
% The Current Maintainer of this work is Haiwen Zhang.
%%

\chapter{公式与符号}

\LaTeX 的公式环境中符号样式符合 \TeX 默认的美国数学学会(AMS)的符号使用习惯,中文论文写作推荐遵循 GB/T 3102.11——1993《物理科学和技术中的数学符号》标准。这里我们给出一些 \LaTeX 中常用的符号表示。


\section{\LaTeX 数学公式模式}

\LaTeX 提供了两种数学公示的写作模式:内联模式和独显模式:

\begin{itemize}
    \item \textbf{内联模式}(inline mode),又称为行内模式,随文模式,将公式显示为段落的一部分。
    \item \textbf{独显模式}(display mode),又称为行间模式,将公式用独立行展示出来,不再作为段落的一部分。
\end{itemize}

\subsection{内联模式}

% TODO

键入如下定义符之一在段落中来使用内联模式书写数学公式符号:

\begin{itemize}
    \item \verb|\(...\)|
    \item \verb|$...$|
    \item \verb|\begin{math}...\end{math}|
\end{itemize}

\subsection{独显模式}

使用如下方式以独显模式表示数学公式:

\begin{itemize}
    \item \verb|\[...\]|
    \item \verb|\begin{displaymath}...\end{displaymath}|
    \item \verb|\begin{equation}...\end{equation}|
\end{itemize}

\textbf{公式插入示例如公式(\ref{E.example})所示。}

\begin{equation}
\gamma_{x}=
\left\{
  \begin{array}{lr}
  0, & {\rm if}~~\;|x| \leq \delta \\
  x, & {\rm otherwise}
  \end{array}
\right.
\label{E.example}
\end{equation}


