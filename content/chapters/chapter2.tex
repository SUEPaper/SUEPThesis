%%
% The BIThesis Template for Bachelor Graduation Thesis
%
% 上海电力大学毕业设计(论文)中英文摘要 —— 使用 XeLaTeX 编译
%
% Copyright 2020-2023 SUEPaper
%
% This work may be distributed and/or modified under the
% conditions of the LaTeX Project Public License, either version 1.3
% of this license or (at your option) any later version.
% The latest version of this license is in
%   http://www.latex-project.org/lppl.txt
% and version 1.3 or later is part of all distributions of LaTeX
% version 2005/12/01 or later.
%
% This work has the LPPL maintenance status `maintained'.
%
% The Current Maintainer of this work is Haiwen Zhang.
%%

\chapter{图表示例}

\section{图片与布局}

\subsection{插图}

图片可以通过\verb|includegraphics|指令插入,我们建议模板使用者将文章所需插入的图片源文件放置在 images 目录中,
另外,矢量图片应使用PDF格式,位图照片则应使用JPG格式(LaTeX不支持TIFF格式)。具有透明背景的栅格图可以使用PNG格式。
模板已经配置了相对路径,所以在文中插图片时,直接用 images 目录下的相对路径即可,比如 images/logo.png ,
在正文中插入只需要\verb|includegraphics{logo.png}|,不需要再增加前缀。

下面是一个简单的插图示例。

\begin{figure}[hbt]
    \centering
    \includegraphics[width=0.3\linewidth]{logo.png}
    \caption{插图示例}
    \label{f.example}
\end{figure}


如果一个图由多个分图(子图)组成,应通过(a),(b),(c)进行标识并附注在分图(子图下方)。

\subsection{横向布局}

模板提供常见的图片布局,比如单图布局\ref{f.example},另外还有横排布局如下:

\begin{figure}[!htb]
    \centering
    \begin{subfigure}[t]{0.24\linewidth}
        \captionsetup{justification=centering}
        \begin{minipage}[b]{1\linewidth}
        \includegraphics[width=1\linewidth]{logo.png}
        \caption{test}
        \end{minipage}
    \end{subfigure}
    \begin{subfigure}[t]{0.24\linewidth}
        \captionsetup{justification=centering}
        \begin{minipage}[b]{1\linewidth}
        \includegraphics[width=1\linewidth]{logo_black.png}
        \caption{test}
        \end{minipage}
    \end{subfigure}
    \begin{subfigure}[t]{0.24\linewidth}
        \captionsetup{justification=centering}
        \begin{minipage}[b]{1\linewidth}
        \includegraphics[width=1\linewidth]{logo.png}
        \caption{test}
        \end{minipage}
    \end{subfigure}
    \begin{subfigure}[t]{0.24\linewidth}
        \captionsetup{justification=centering}
        \begin{minipage}[b]{1\linewidth}
        \includegraphics[width=1\linewidth]{logo_black.png}
        \caption{test}
        \end{minipage}
    \end{subfigure}
    \caption{图片横排布局示例}
    \label{f.row}
\end{figure}

\subsection{纵向布局}

纵向布局如图\ref{f.col}

\begin{figure}[!htb]
    \centering
    \begin{subfigure}[t]{0.15\linewidth}
        \captionsetup{justification=centering} %ugly hacks
        \begin{minipage}[b]{1\linewidth}
        \includegraphics[width=1\linewidth]{logo.png}
        \caption{test}
        \end{minipage}
    \end{subfigure}\\
    \begin{subfigure}[t]{0.15\linewidth}
        \captionsetup{justification=centering} %ugly hacks
        \begin{minipage}[b]{1\linewidth}
        \includegraphics[width=1\linewidth]{logo_black.png}
        \caption{test}
        \end{minipage}
    \end{subfigure}
    \caption{图片纵向布局示例}
    \label{f.col}
\end{figure}

\subsection{竖排多图横排布局}

\begin{figure}[!htb]
    \centering
    \begin{subfigure}[t]{0.13\linewidth}
        \captionsetup{justification=centering} 
        \begin{minipage}[b]{1\linewidth}
        \includegraphics[width=1\linewidth]{logo.png} 
        \vspace{-1ex} \vfill
        \includegraphics[width=1\linewidth]{logo_black.png}
        \caption{aaa}
        \end{minipage}
    \end{subfigure}
    \begin{subfigure}[t]{0.13\linewidth}
        \captionsetup{justification=centering} 
        \begin{minipage}[b]{1\linewidth}
        \includegraphics[width=1\linewidth]{logo_black.png} 
        \vspace{-1ex} \vfill
        \includegraphics[width=1\linewidth]{logo.png}
        \caption{bbb}
        \end{minipage}
    \end{subfigure}
    \caption{图片竖排多图横排布局}
    \label{f.suep_col_row}
\end{figure}

竖排多图横排布局如图\ref{f.suep_col_row}所示。注意看(a)、(b)编号与图关系


\subsection{横排多图竖排布局}

\begin{figure}[!htb]
    \centering
    \begin{subfigure}[t]{0.3\linewidth}
        \captionsetup{justification=centering} 
        \begin{minipage}[b]{1\linewidth}
        \includegraphics[width=0.45\linewidth]{logo.png}
        \includegraphics[width=0.45\linewidth]{logo_black.png}
        \caption{}
        \end{minipage}
    \end{subfigure}\\
    \begin{subfigure}[t]{0.3\linewidth}
        \captionsetup{justification=centering} 
        \begin{minipage}[b]{1\linewidth}
        \includegraphics[width=0.45\linewidth]{logo_black.png}
        \includegraphics[width=0.45\linewidth]{logo.png}
        \caption{}
        \end{minipage}
    \end{subfigure}
    \caption{图片横排多图竖排布局}
    \label{f.suep_row_col}
\end{figure}

横排多图竖排布局如图\ref{f.suep_row_col}所示。注意看(a)、(b)编号与图关系。

\subsection{2x2图片布局}

\begin{figure}[!htb]
    \centering
    \begin{subfigure}[t]{0.3\linewidth}
        \captionsetup{justification=centering}
        \begin{minipage}[b]{1\linewidth}
            \centering
            \includegraphics[width=0.45\linewidth]{logo.png}
            \caption{}
        \end{minipage}
    \end{subfigure}
    \hspace{-5em}
    \begin{subfigure}[t]{0.3\linewidth}
        \captionsetup{justification=centering}
        \begin{minipage}[b]{1\linewidth}
            \centering
            \includegraphics[width=0.45\linewidth]{logo_black.png}
            \caption{}
        \end{minipage}
    \end{subfigure}\\
    \begin{subfigure}[t]{0.3\linewidth}
        \captionsetup{justification=centering}
        \begin{minipage}[b]{1\linewidth}
            \centering
            \includegraphics[width=0.45\linewidth]{logo.png}
            \caption{}
        \end{minipage}
    \end{subfigure}
    \hspace{-5em}
    \begin{subfigure}[t]{0.3\linewidth}
        \captionsetup{justification=centering}
        \begin{minipage}[b]{1\linewidth}
            \centering
            \includegraphics[width=0.45\linewidth]{logo_black.png}
            \caption{}
        \end{minipage}
    \end{subfigure}
    \caption{图片2x2布局}
    \label{f.csu_2x2}
\end{figure}

\newpage

\section{图表编号}

本节主要阐述在图表下侧的编号,以及引用时的编号设置问题。
使用在上一节(上一个section, \verb|\section{插图}| )中引入的图\ref{f.example},
以及在本节加入下方新图片对比来说明。

\begin{figure}[hbt]
    \centering
    \includegraphics[width=0.3\linewidth]{logo.png}
    \caption{图标编号示例}
    \label{f.example.2}
\end{figure}

可以看到在上一节“插图示例”的编号为图\ref{f.example}。
而在本节“图标编号示例”为图\ref{f.example.2}。


上海电力大学(Shanghai University of Electric Power),位于上海市,是中央与上海市共建、
以上海市管理为主的全日制普通高等院校,是教育部首批“卓越工程师教育培养计划”试点院校、
上海高水平地方应用型高校建设单位、上海市首批深化创新创业教育改革示范高校,
为国际电力高校联盟永久理事长单位、全球能源互联网发展合作组织会员单位、中国电力高校联盟成员单位、
一带一路电力高校联盟和一带一路电力产学研联盟发起成员单位,入选国家级新工科研究与实践项目、
国家级大学生创新创业训练计划、上海高校知识服务能力提升工程、上海高等学校一流本科建设引领计划、
上海高等学校一流研究生教育引领计划、上海市级新工科研究与改革实践项目。

学校创建于1951年,长期隶属于国家电力部门管理;1985年更名为上海电力学院,开始本科层次办学;
2000年划归上海市管理。 学校历经了上海电业学校、上海动力学校、上海电力学校、上海电力高等专科学校、上海电力学院的发展演变;
2006年正式开始硕士层次办学,2018年成为博士学位授予单位。
2018年11月30日,上海电力学院正式更名为上海电力大学。

截至2021年11月,学校有杨浦、浦东两个校区,占地面积近1200亩;全日制在校生12000余人;在编教职工1100余人;
设有13个二级院部和38个本科专业;有1个一级学科博士学位授权点,7个一级学科硕士学位授权点 ,
4个硕士专业学位授权点;拥有国家大学科技园、国家级技术转移中心及11个省部级以上科研平台。 