%%
% The SUEPThesis Template for Bachelor Graduation Thesis
%
% 上海电力大学毕业设计(论文)中英文摘要 —— 使用 XeLaTeX 编译
%
% Copyright 2020-2023 SUEPaper
%
% This work may be distributed and/or modified under the
% conditions of the LaTeX Project Public License, either version 1.3
% of this license or (at your option) any later version.
% The latest version of this license is in
%   http://www.latex-project.org/lppl.txt
% and version 1.3 or later is part of all distributions of LaTeX
% version 2005/12/01 or later.
%
% This work has the LPPL maintenance status `maintained'.
%
% The Current Maintainer of this work is Haiwen Zhang.
%%

\chapter{不允许缺货的WPCR生产和存储模型}

\section{不允许缺货的WPCR生产和存储模型的建立}

下面研究$n$阶段生产存储计划,$d_k$设为第$k$阶段对产品WPCR的需求,根据题目设置如下约定:
\begin{equation}
    C_j(x_k)=k_j + c_j \cdot x_k
\end{equation}
\begin{equation}
    T_k=\sum_{j=1}t_j \cdot x_k
\end{equation}
\begin{equation}
    t_k \leq T_k
\end{equation}

$x_k$表示第$k$天的产量,$c_j$表示第$j$种单间库存费用,$k_j$表示第$j$种生产准备费用,$C_j$表示第$j$种成本费用,
$t_j$表示第$j$种单件消耗工时,$t_k$表示第$k$天生产所有组件所耗时间,
$T_k$表示第$k$天生产所有组件总工时限制\cite{芮世春2012生产与存储问题}。

\subsection{生产和库存平衡关系}

设为$x_k$第$k$阶段WPCR的生产量,$v_k$为第$k$阶段结束时WPCR的库存量,则对于WPCR而言,有
\begin{equation}
    v_k = v_{k-1} + x_k - d_k
\end{equation}

再结合该工厂第一天(周一)开始时没有任何组件库存,也不希望第7天(周日)结束后留下任何组件库存,则
\begin{equation}
    \begin{cases}
        v_0=0 \\ 
        v_k=v_{k-1} + x_k - d_k,k=1,2,...,n \\ 
        v_n=0
    \end{cases}
\end{equation}

某工厂生产的WPCR装置需要用3个容器艇(用A表示)、4个机器臂(用B表示)、5个动力系统(用C表示)组装而成。

对A而言,第$k$阶段生产量为$x_k^A$,第$k$阶段结束时A的库存量为$v_k^A$,消耗量为$x_k^A=3x_k$,
则根据A的生产和消耗平衡关系,有
\begin{equation}
    v_k^A=v_{k-1}^A + x_k^A - 3x_k
\end{equation}

对B而言,第$k$阶段生产量为$x_k^B$,第$k$阶段结束时B的库存量为$v_k^B$,消耗量为$x_k^B=4x_k$,
则根据B的生产和消耗平衡关系,有
\begin{equation}
    v_k^B=v_{k-1}^B + x_k^B - 4x_k
\end{equation}

对C而言,第k阶段生产量为$x_k^C$,第k阶段结束时C的库存量为$v_k^C$,消耗量为$x_k^C=5x_k$,则根据C的生产和消耗平衡关系,有
\begin{equation}
    v_k^C=v_{k-1}^C + x_k^C - 5x_k
\end{equation}

同理,对于A1、A2、A3、B1、B2、C1、C2、C3可以类似研究,设生产一件WPCR所需要的A1、A2、A3、B1、B2、C1、C2、C3分别为
$u_{A1}$,$u_{A2}$,$u_{A3}$,$u_{B1}$,$u_{B2}$,$u_{C1}$,$u_{C2}$,$u_{C3}$,则
\begin{equation}
    v_k^{Ai}=v_{k-1}^{Ai} + x_k^{Ai} - u_{Ai}x_k,i=1,2,3
\end{equation}
\begin{equation}
    v_k^{Bi}=v_{k-1}^{Bi} + x_k^{Bi} - u_{Bi}x_k,i=1,2
\end{equation}
\begin{equation}
    v_k^{Ci}=v_{k-1}^{Ci} + x_k^{Ci} - u_{Ci}x_k,i=1,2
\end{equation}

具体地,有
\begin{equation}
    \begin{cases}
        v_k^{A1}=v_{k-1}^{A1} + x_k^{A1} - 18x_k \\ 
        v_k^{A2}=v_{k-1}^{A2} + x_k^{A2} - 24x_k \\ 
        v_k^{A3}=v_{k-1}^{A3} + x_k^{A3} - 6x_k \\ 
        v_k^{B1}=v_{k-1}^{B1} + x_k^{B1} - 8x_k \\ 
        v_k^{B2}=v_{k-1}^{B2} + x_k^{B2} - 16x_k \\ 
        v_k^{C1}=v_{k-1}^{C1} + x_k^{C1} - 40x_k \\ 
        v_k^{C2}=v_{k-1}^{C2} + x_k^{C2} - 10x_k \\ 
        v_k^{C3}=v_{k-1}^{C3} + x_k^{C3} - 60x_k
    \end{cases}
\end{equation}

\subsection{不允许缺货约束}

按照工厂的信誉要求,目前接收的所有订单到期必须全部交货,轻易不能有缺货事件发生。
引入不允许缺货模型,或者设缺货损失为无穷大,则有如下条件限制。
\begin{equation}
    s.t.
    \begin{cases}
        v_k \geqslant 0, V_k^A \geqslant 0,V_k^B \geqslant 0,V_k^C \geqslant 0, & k=1,2,...,n \\
        v_k^{A1} \geqslant 0, V_k^{A2} \geqslant 0,V_k^{A3} \geqslant 0, & k=1,2,...,n \\
        v_k^{B1} \geqslant 0, V_k^{B2} \geqslant 0, & k=1,2,...,n \\
        v_k^{C1} \geqslant 0, V_k^{C2} \geqslant 0,V_k^{C3} \geqslant 0, & k=1,2,...,n \\
    \end{cases}
\end{equation}

\subsection{总工时约束}

已知A、B、C的工时消耗分别为3时/件、5时/件和5时/件,即生产1件A需要占用3个工时,
生产1件B需要占用5个工时,生产1件C需要占用5个工时。

第$k$阶段$x_k^A$、$x_k^A$、$x_k^A$的生产量为分别为,所需总工时不能超过总工时$T_k$的限制,
即工时约束条件为
\begin{equation}
    3x_k^A + 5x_k^B + 5x_k^C \leq T_k
\end{equation}

\subsection{WPCR各组件的生产准备成本和存储成本}

为了顺利生产WPCR,工厂在某一天生产组件产品时,需要付出一个与生产数量无关的固定成本,称为生产准备费用。
比如第一天生产了A,则要支付A的生产准备费用,若第二天再生产A,则需要再支付A的生产准备费用。

首先根据表\ref{t.ch2-1}可知,单件WPCR生产准备费用为240元,单件WPCR库存费用为5元。
生产组件A的生产准备费用为120元,库存费用为2元,组件B的生产准备费用为160元,
库存费用为1.5元,组件C的生产准备费用为180元,库存费用为1.7元。
生产零件A1的生产准备费用为40元,库存费用为5元,生产零件A2的生产准备费用为60元,
库存费用为3元,生产零件A3的生产准备费用为50元,库存费用为6元。
生产零件B1的生产准备费用为80元,库存费用为4元,生产零件B2的生产准备费用为100元,库存费用为5元。
生产零件C1的生产准备费用为60元,库存费用为3元,生产零件C2的生产准备费用为40元,库存费用为2元,
生产零件C3的生产准备费用为70元,库存费用为3元。

\begin{table}[h]
    \renewcommand\arraystretch{1.5} 
    \centering
    \caption{每次生产准备费用和单件库存费用(单位:元)}
    \xiaowu
    \label{t.ch2-1}
    \begin{tabular}{|c|c|c|c|c|c|c|c|c|c|c|c|c|}
    \hline
     \textbf{产品} & \textbf{WPCR} & \textbf{A} & \textbf{A1} & \textbf{A2} & \textbf{A3} & \textbf{B} & \textbf{B1} & \textbf{B2} & \textbf{C} & \textbf{C1} & \textbf{C2} & \textbf{C3} \\ \hline
     \textbf{生产准备费用} &  &  &  &  &  &  &  &  &  &  &  &  \\ \hline
     \textbf{单件库存费用} &  &  &  &  &  &  &  &  &  &  &  &  \\ \hline
    \end{tabular}
\end{table}



设$c_k(x_k)$表示第$k$阶段生产$x_k$的WPCR时的生产准备费用,则
\begin{equation}
    c_k(x_k)=\begin{cases}
        0, & x_k=0\\
        240, & x_k>0
    \end{cases}
\end{equation}

引入$0-1$变量$y_k$和任意大正数$M$(取相对$10^3$倍的数)
\begin{equation}
    y_k=\begin{cases}
        1, & x_k>0,\text{表示生产}x_k\text{生产准备费用不为}0 \\
        0, & x_k=0,\text{表示不生产}x_k\text{生产准备费用为}0
    \end{cases}
\end{equation}
且满足
\begin{equation}
    x_k \leq My_k
\end{equation}

则生产准备费用表达式等价于
\begin{equation}
    c_k(x_k)=240y_k
\end{equation}
注意到$v_k$为第$k$阶段结束时WPCR的库存量,单件WPCR的库存费用为5元/件,则库存费用为
\begin{equation}
    h_k(v_k)=5v_k
\end{equation}

这样对于WPCR的费用为
\begin{equation}
    C_{\uppercase\expandafter{\romannumeral1}}=\sum_{k=1}^{n}[C_k(x_k)+h_k(v_k)]=\sum_{k=1}^{n}(240y_k+5v_k)
\end{equation}

同理研究A、B、C的生产准备费和库存费用。引入$0-1$变量$y_k^A$,$y_k^B$,$y_k^C$,即
\begin{equation}
    y_k^A=\begin{cases}
        1, & x_k^A>0\\
        0, & x_k^A=0
    \end{cases}
\end{equation}
\begin{equation}
    y_k^B=\begin{cases}
        1, & x_k^B>0\\
        0, & x_k^B=0
    \end{cases}
\end{equation}
\begin{equation}
    y_k^C=\begin{cases}
        1, & x_k^C>0\\
        0, & x_k^C=0
    \end{cases}
\end{equation}
且满足
\begin{equation}
    x_k^A \leq My_k^A,x_k^B \leq My_k^B,x_k^C \leq My_k^C
\end{equation}
则对于A、B、C的生产和存储费用为
\begin{equation}
    C_{\uppercase\expandafter{\romannumeral2}}=\sum_{k=1}^{n}(120y_k^A+2v_k^A+160y_k^B+1.5v_k^B+180y_k^C+1.7v_k^C)
\end{equation}

再进一步研究A1、A2、A3、B1、B2、C1、C2、C3的生产准备费和库存费用。
引入$0-1$变量$y_k^{A1}$,$y_k^{A2}$,$y_k^{A3}$,$y_k^{B1}$,$y_k^{B2}$,$y_k^{C1}$,$y_k^{C2}$,$y_k^{C3}$即
\begin{equation}
    y_k^{Ai}=\begin{cases}
        1, & x_k^{Ai}>0\\
        0, & x_k^{Ai}=0
    \end{cases},i=1,2,3
\end{equation}
\begin{equation}
    y_k^{Bi}=\begin{cases}
        1, & x_k^{Bi}>0\\
        0, & x_k^{Bi}=0
    \end{cases},i=1,2
\end{equation}
\begin{equation}
    y_k^{Ci}=\begin{cases}
        1, & x_k^{Ci}>0\\
        0, & x_k^{Ci}=0
    \end{cases},i=1,2,3
\end{equation}
且满足
\begin{equation}
    \begin{cases}
        x_k^{Ai} \leq My_k^{Ai}, &i=1,2,3 \\
        x_k^{Bi} \leq My_k^{Bi},&i=1,2 \\
        x_k^{Ci} \leq My_k^{Ci},&i=1,2,3
    \end{cases}
\end{equation}
则对于A1、A2、A3、B1、B2、C1、C2、C3的生产和存储费用为
\begin{equation}
    \begin{aligned}
        C_{\uppercase\expandafter{\romannumeral3}} &= 
            \sum_{k=1}^{n}(40y_k^{A1}+5v_k^{A1}+60y_k^{A2}+3v_k^{A2}+50y_k^{A3}+6v_k^{A3}) \\
            &+ \sum_{k=1}^{n}(80y_k^{B1}+4v_k^{B1}+100y_k^{B2}+5v_k^{B2}) \\
            &+ \sum_{k=1}^{n}(60y_k^{C1}+3v_k^{C1}+40y_k^{C2}+2v_k^{C2}+70y_k^{C3}+3v_k^{C3})
    \end{aligned}
\end{equation}



\subsection{不允许缺货的WPCR生产和存储模型}

汇总1-20,1-25,1-30三式得目标函数为
\begin{equation}
    \text{min}\sum_{k=1}^{n}(C_{\uppercase\expandafter{\romannumeral1}}
        + C_{\uppercase\expandafter{\romannumeral2}} 
        + C_{\uppercase\expandafter{\romannumeral3}})
\end{equation}
综上,所有限制条件总结如下:
\begin{equation}
    s.t.
    \begin{cases}
        v_k=v_{k-1}+x_k-d_k,v_0=0 \\
        3x_k^A+5x_k^B+5x_k^C \leq T_k \\
        ... \\
        y_k,y_k^A,y_k^B,y_k^C,y_k^{Ai},y_k^{Bi},y_k^{Ci}=0\text{或}1
    \end{cases} \qquad k=1,2,...,n
\end{equation}

\section{不允许缺货情形WPCR生产和存储模型的求解}
表2-2为每天WPCR需求和关键设备工时限制,可以看出周一到周日WPCR的需求分别为39、36、38、40、37、33、40,
同时A、B、C周一到周日生产总工时限制分别为4500、2500、2750、2100、2500、2750、1500。
通过编写Lingo程序求解结果为表2-3所示。

从表2-2可以看出,周一没有库存费用,只有生产准备费用。周二只有关于B组件的组装计划,而周三只有组件B不需要组装。
周四和周日没有生产计划,只有库存费用。周五和周六的生产计划比较全面,生产准备费用和库存费用都需要。
同时,有该表可知,如果生产计划很全面,WPCR、A、B、C都需要组装,无论其中需要组装的数量是多少,
生产准备费用都不变,都为1200元。总的生产和存储成本为6260.9元。