%%
% The SUEPThesis Template for Bachelor Graduation Thesis
%
% 上海电力大学毕业设计(论文)中英文摘要 —— 使用 XeLaTeX 编译
%
% Copyright 2020-2023 SUEPaper
%
% This work may be distributed and/or modified under the
% conditions of the LaTeX Project Public License, either version 1.3
% of this license or (at your option) any later version.
% The latest version of this license is in
%   http://www.latex-project.org/lppl.txt
% and version 1.3 or later is part of all distributions of LaTeX
% version 2005/12/01 or later.
%
% This work has the LPPL maintenance status `maintained'.
%
% The Current Maintainer of this work is Haiwen Zhang.
%%

\chapter{连续多周生产统筹规划的WPCR生产和存储模型}

\section{连续多周生产统筹规划模型的建立}

令$V_{i-1}$为状态变量,表示第$i$天开始时的库存量;$x_i$为决策变量,表示第$i$天的生产量。
状态转移方程为
\begin{equation*}
    v_i=v_{i-1}+x_i-d_i \qquad i=1,2,...,n
\end{equation*}

最优值函数$f_i(v_i)$表示从第一天初始库存量为$0$到第$i$天末库存量为$v_i$时的最小总费用,
因而可写出顺序递推关系式\cite{韩中庚2007实用运筹学模型}:
\begin{equation}
    f_i(v_i)=\mathop{\min}_{0\leq x_i \leq \sigma_i}[C_i(x_i)+h_i(v_i)+f_{i-1}(V_{i-1})] \qquad i=1,2,...,n
\end{equation}

其中,
\begin{equation}
    \sigma_i=\min(v_i+d_i,m)
\end{equation}

不允许缺货的WPCR生产和存储模型在无存货无结余的情况下进行一周(7天)的机器人的组装计划,
求解总成本最小。

然而,事实上,组件A、B、C需要提前一天生产入库才能组装WPCR,
A1、A2、A3、B1、B2、C1、C2、C3也需要提前一天生产入库才能组装A、B、C。
在连续多周生产情况下,需要统筹规划。
由不允许缺货的WPCR生产和存储模型的求解结果列出库存量和生产准备成本如下:
\begin{equation}
    v_k=v_{k-1}+x_k-d_k
\end{equation}
\begin{equation}
    C_k(x_k)=\begin{cases}
        0, & x_k=0 \\
        K+c_j \cdot x_k, & x_k=1,2,...,n \\
        \infty, & x_k > m
    \end{cases}
\end{equation}

$d_k$表示第$k$阶段对组件的需求量,$v_k$表示第$k$阶段结束时组件库存量,$m$表示每个阶段最多能生产该组件的上限数。
\begin{equation}
    \min g = \sum_{n}^{k=1}[C_k(x_k)+h_k(v_k)]
\end{equation}
\begin{equation}
    \begin{cases}
        v_0=0,v_n=0 \\
        v_k=\sum_{k}^{i}(x_i - d_i) \ge 0 & k=2,3,...,n-1 \\
        0 \leq x_k \leq m & k=1,2,...,n \\
        x_k \text{为整数} & k=1,2,...,n
    \end{cases}
\end{equation}

$h_k(v_k)$表示第$k$阶段结束时有库存量$v_k$所需的存储费用。
\begin{equation}
    y_k=\begin{cases}
        1, & x_k > 0, \text{表示生产}x_k,\text{生产准备费不为}0 \\
        0, & x_k = 0, \text{表示不生产}x_k,\text{生产准备用费为}0 
    \end{cases}
\end{equation}
\begin{equation}
    C=\sum_{n}^{k=1}[C_k(x_k)+h_k{v_k}]=\sum_{k=1}^{n}[K \cdot y_k + c_j \cdot x_k]
\end{equation}

对于WPCR而言,
\begin{equation}
    v_k=v_{k-1} + x_k - d_k    
\end{equation}

对于A,B,C组件而言,根据A,B,C生产与组装关系,有
\begin{equation}
    v_{k+2}^A=v_{k+1}^A + x_{k+2}^A - 3x_{k+2}
\end{equation}
\begin{equation}
    v_{k+2}^B=v_{k+1}^B + x_{k+2}^B - 4x_{k+2}
\end{equation}
\begin{equation}
    v_{k+2}^C=v_{k+1}^C + x_{k+2}^C - 5x_{k+2}
\end{equation}

同理,对于A1、A2、A3、B1、B2、C 1、C2、C3可以类似研究,设生产一件WPCR所需要的
A1、A2、A3、B1、B2、C1、C2、C3分别为$u_{A1}$,$u_{A2}$,$u_{A3}$,$u_{B1}$,
$u_{B2}$,$u_{C1}$,$u_{C2}$,$u_{C3}$,则
\begin{equation}
    v_{k+1}^{Ai}=v_k^{Ai} + x_{k+1}^{Ai} - u_{Ai}3x_k,\quad i=1,2,3
\end{equation}
\begin{equation}
    v_{k+1}^{Bi}=v_k^{Bi} + x_{k+1}^{Bi} - u_{Bi}3x_k,\quad i=1,2
\end{equation}
\begin{equation}
    v_{k+1}^{Ci}=v_k^{Ci} + x_{k+1}^{Ci} - u_{Ci}3x_k,\quad i=1,2,3
\end{equation}

$s.t.$
\begin{equation}
    3x_{k+2}^A+5x_{k+2}^B+5x_{k+2}^C \leq T_k
\end{equation}
\begin{equation}
    y_k=\begin{cases}
        1, & x_k > 0, \text{表示生产}x_k,\text{生产准备费不为}0 \\
        0, & x_k = 0, \text{表示不生产}x_k,\text{生产准备用费为}0 
    \end{cases}
\end{equation}
\begin{equation}
    x_k \leq My_k
\end{equation}

则各费用为,
\begin{equation}
    C_{\uppercase\expandafter{\romannumeral1}}^{'}=
        \sum_{n}^{k=1}[C_k(x_k)+h_k{v_k}]=
        \sum_{n}^{k=1}(240y_k+5v_k)
\end{equation}
\begin{equation}
    C_{\uppercase\expandafter{\romannumeral2}}^{'}=
        \sum_{n}^{k=1}(120y_{k+2}^A+2v_{k+2}^A+160y_{k+2}^B+1.5v_{k+2}^B+180y_{k+2}^C+1.7v_{k+2}^C)
\end{equation}

\begin{equation}
    \begin{aligned}
        C_{\uppercase\expandafter{\romannumeral3}}^{'} &= 
            \sum_{k=1}^{n}(40y_{k+1}^{A1}+5v_{k+1}^{A1}+60y_{k+1}^{A2}+3v_{k+1}^{A2}+50y_{k+1}^{A3}+6v_{k+1}^{A3}) \\
            &+ \sum_{k=1}^{n}(80y_{k+1}^{B1}+4v_{k+1}^{B1}+100y_{k+1}^{B2}+5v_{k+1}^{B2}) \\
            &+ \sum_{k=1}^{n}(60y_{k+1}^{C1}+3v_{k+1}^{C1}+40y_{k+1}^{C2}+2v_{k+1}^{C2}+70y_{k+1}^{C3}+3v_{k+1}^{C3})
    \end{aligned}
\end{equation}



汇总三式得目标函数为
\begin{equation}
    \min\sum_{k=1}^{n}(
            C_{\uppercase\expandafter{\romannumeral1}}^{'}+
            C_{\uppercase\expandafter{\romannumeral2}}^{'}+
            C_{\uppercase\expandafter{\romannumeral3}}^{'}) 
\end{equation}

\section{连续多周生产统筹规划模型的求解}

基于以下表\ref{T.ch3-1}的连续30周的WPCR需求的数据,
带入到连续多周生产统筹规划模型中并求解,得出表\ref{T.ch3-2}的求解结果表格。
\begin{table}[H]
    \renewcommand\arraystretch{1.5} 
    \centering
    \caption{考虑检修的生产和存储模型求解结果展示}
    \label{T.ch3-1}
    \xiaowu
    \begin{tabular}{|c|c|c|c|c|c|c|c|}
    \hline
    \textbf{天} & \textbf{周一} & \textbf{周二} & \textbf{周三} & \textbf{周四} & \textbf{周五} & \textbf{周六} & \textbf{周日} \\ \hline
    \textbf{第1周}  & 1 & 2 & 3 & 4 & 5 & 6 & 7 \\ \hline
    \textbf{第2周}  & 1 & 2 & 3 & 4 & 5 & 6 & 7 \\ \hline
    \textbf{第3周}  & 1 & 2 & 3 & 4 & 5 & 6 & 7 \\ \hline
    \textbf{第4周}  & 1 & 2 & 3 & 4 & 5 & 6 & 7 \\ \hline
    \textbf{第5周}  & 1 & 2 & 3 & 4 & 5 & 6 & 7 \\ \hline
    \textbf{第6周}  & 1 & 2 & 3 & 4 & 5 & 6 & 7 \\ \hline
    \textbf{第7周}  & 1 & 2 & 3 & 4 & 5 & 6 & 7 \\ \hline
    \textbf{第8周}  & 1 & 2 & 3 & 4 & 5 & 6 & 7 \\ \hline
    \textbf{第9周}  & 1 & 2 & 3 & 4 & 5 & 6 & 7 \\ \hline
    \textbf{第10周}  & 1 & 2 & 3 & 4 & 5 & 6 & 7 \\ \hline
    \textbf{第11周}  & 1 & 2 & 3 & 4 & 5 & 6 & 7 \\ \hline
    \textbf{第12周}  & 1 & 2 & 3 & 4 & 5 & 6 & 7 \\ \hline
    \textbf{第13周}  & 1 & 2 & 3 & 4 & 5 & 6 & 7 \\ \hline
    \textbf{第14周}  & 1 & 2 & 3 & 4 & 5 & 6 & 7 \\ \hline
    \textbf{第15周}  & 1 & 2 & 3 & 4 & 5 & 6 & 7 \\ \hline
    \textbf{第16周}  & 1 & 2 & 3 & 4 & 5 & 6 & 7 \\ \hline
    \textbf{第17周}  & 1 & 2 & 3 & 4 & 5 & 6 & 7 \\ \hline
    \textbf{第18周}  & 1 & 2 & 3 & 4 & 5 & 6 & 7 \\ \hline
    \textbf{第19周}  & 1 & 2 & 3 & 4 & 5 & 6 & 7 \\ \hline
    \textbf{第20周}  & 1 & 2 & 3 & 4 & 5 & 6 & 7 \\ \hline
    \textbf{第21周}  & 1 & 2 & 3 & 4 & 5 & 6 & 7 \\ \hline
    \textbf{第22周}  & 1 & 2 & 3 & 4 & 5 & 6 & 7 \\ \hline
    \textbf{第23周}  & 1 & 2 & 3 & 4 & 5 & 6 & 7 \\ \hline
    \textbf{第24周}  & 1 & 2 & 3 & 4 & 5 & 6 & 7 \\ \hline
    \textbf{第25周}  & 1 & 2 & 3 & 4 & 5 & 6 & 7 \\ \hline
    \textbf{第26周}  & 1 & 2 & 3 & 4 & 5 & 6 & 7 \\ \hline
    \textbf{第27周}  & 1 & 2 & 3 & 4 & 5 & 6 & 7 \\ \hline
    \textbf{第28周}  & 1 & 2 & 3 & 4 & 5 & 6 & 7 \\ \hline
    \textbf{第29周}  & 1 & 2 & 3 & 4 & 5 & 6 & 7 \\ \hline
    \textbf{第30周}  & 1 & 2 & 3 & 4 & 5 & 6 & 7 \\ \hline
    \end{tabular}
\end{table}

\begin{table}[H]
    \centering
    \renewcommand\arraystretch{1.5} 
    \caption{连续多周生产统筹规划的WPCR生产和存储模型求解的结果}
    \xiaowu
    \label{T.ch3-2}
    \begin{tabular}{|c|c|c|c|c|c|c|}
        \hline
        \textbf{日期} & \textbf{WPCR组装数量} & \textbf{A组装数量} & \textbf{B组装数量} & \textbf{C组装数量} & \textbf{生产准备费用} & \textbf{库存费用} \\ \hline
        \textbf{周一} & 237 & 316 & 395 & 41 & 700 & 1619.5 \\ \hline
        \textbf{周二} & 237 & 316 & 395 & 41 & 700 & 1619.5 \\ \hline
        \textbf{周三} & 237 & 316 & 395 & 41 & 700 & 1619.5 \\ \hline
        \textbf{周四} & 237 & 316 & 395 & 41 & 700 & 1619.5 \\ \hline
        \textbf{周五} & 237 & 316 & 395 & 41 & 700 & 1619.5 \\ \hline
        \textbf{周六} & 237 & 316 & 395 & 41 & 700 & 1619.5 \\ \hline
        \textbf{周日} & 237 & 316 & 395 & 41 & 700 & 1619.5 \\ \hline
        \textbf{总和} & 237 & 316 & 395 & 41 & 700 & 1619.5 \\ \hline 
    \end{tabular}
\end{table}

从表\ref{T.ch3-2}可以看出,连续多周生产统筹规划模型中周一、周二、周五、周六、周日的生产计划很全面,
和不允许缺货的WPCR生产和存储模型一样,如果生产计划很全面,WPCR、A、B、C都需要组装,
无论其中需要组装的数量是多少,生产准备费用都不变,都为1200元。周三只进行WPCR组装的生产计划,
而周四进行组件A、B、C的组装生产计划。总的生产和存储成本为207658元,
连续多周生产统筹规划模型平均单周所用的成本为6921.93元,
要高于不允许缺货的WPCR生产和存储模型每周的生产成本。