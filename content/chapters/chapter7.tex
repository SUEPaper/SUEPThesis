%%
% The SUEPThesis Template for Bachelor Graduation Thesis
%
% 上海电力大学毕业设计(论文)中英文摘要 —— 使用 XeLaTeX 编译
%
% Copyright 2020-2023 SUEPaper
%
% This work may be distributed and/or modified under the
% conditions of the LaTeX Project Public License, either version 1.3
% of this license or (at your option) any later version.
% The latest version of this license is in
%   http://www.latex-project.org/lppl.txt
% and version 1.3 or later is part of all distributions of LaTeX
% version 2005/12/01 or later.
%
% This work has the LPPL maintenance status `maintained'.
%
% The Current Maintainer of this work is Haiwen Zhang.
%%

\chapter{结论}

\section{模型评价}

(1)模型构建了需求与生产的反馈过程及在需求随机的情况下,优化生产组装计划,进一步拓展了模型的适用范围。

(2)考虑到总工时的动态改变,设计了一种带末端优化的遗传算法求解,并构建了一种双层规划模型,通过修改部分参数,
模型能同时适用于多种场景。

(3)模型的自适应能力强,针对未来订单未知情形的WPCR不确定生产和存储模型加入了自适应动态规划,
在遗传训练下具有更高的适应性和推广性,对供求关系动态平衡的研究具有一定的数学理论价值。

\section{模型的缺点}

(1)只考虑了生产总工时,没有考虑有多少生产机器以及如何安排生产顺序,在实际应用中考虑不够全面。

(2)总费用只考虑了生产准备费及存储费用,没有考虑人力、物力、时间等成本。

(3)本文A、B、C消耗工时固定,未考虑组件加工过程中有可能出现的时间差。

\section{模型的改进}

(1)研究在更多约束条件下,完成多任务、多目标的混合优化模型,增加模型的适应性和广谱性,
进而探讨实际生产中供求平衡和成本优化问题。

(2)发展模型的逆向运用,即已知生产成本的前提下,解决具体生产分配问题。