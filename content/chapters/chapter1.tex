%%
% The BIThesis Template for Bachelor Graduation Thesis
%
% 上海电力大学毕业设计(论文)中英文摘要 —— 使用 XeLaTeX 编译
%
% Copyright 2020-2023 SUEPaper
%
% This work may be distributed and/or modified under the
% conditions of the LaTeX Project Public License, either version 1.3
% of this license or (at your option) any later version.
% The latest version of this license is in
%   http://www.latex-project.org/lppl.txt
% and version 1.3 or later is part of all distributions of LaTeX
% version 2005/12/01 or later.
%
% This work has the LPPL maintenance status `maintained'.
%
% The Current Maintainer of this work is Haiwen Zhang.
%%

\chapter{一级标题}

这是上海电力大学本科学位论文\LaTeX{}模板,下面的文字主要作用为对重构后的模板样式设置进行测试。
测试样例基本覆盖模板设定,包括多级标题的基本样式,段落与缩进距离。

\section{二级标题}

\subsection{三级标题}

一级标题根据学校提供的Word模板要求,小三号黑体居中,章节号空一个汉字,段前0行,段后0.5行
并且每一章节单独起一页,章节号格式应使用中文汉字。

二级标题为黑体四号字,居左,1.5倍行距,段前0.5行,段后0行。章节号后空一个汉字。

三级标题黑体小四号字,1.5倍行距,段前0.5行,段后0行,缩进两个汉字,章节号后空一个汉字。

所有标题样式由 \textbf{suepthesis.cls} 模板文件 \textbf{ctexset} 进行设置。

\section{字体}

正文字体默认使用小四号宋体,英文为小四号 Times New Romen,各段行首缩进两个汉字



英文字体展示如下:

TeX (/tɛx, tɛk/, see below), stylized within the system as TEX, is a typesetting system (or a "formatting system") which was designed and mostly written by Donald Knuth\cite{knuth1984texbook} and released in 1978. TeX is a popular means of typesetting complex mathematical formulae; it has been noted as one of the most sophisticated digital typographical systems.


\subsection{调节字号}

可以使用 命令来调节字号。

\begin{tabular}{ll}
  \verb|\chuhao | & \chuhao  初号字 English \\
  \verb|\xiaochu| & \xiaochu 小初号 English \\
  \verb|\yihao  | & \yihao  一号字 English \\
  \verb|\xiaoyi | & \xiaoyi 小一号 English \\
  \verb|\erhao  | & \erhao  二号字 English \\
  \verb|\xiaoer | & \xiaoer 小二号 English \\
  \verb|\sanhao | & \sanhao  三号字 English \\
  \verb|\xiaosan| & \xiaosan 小三号 English \\
  \verb|\sihao  | & \sihao  四号字 English \\
  \verb|\xiaosi | & \xiaosi 小四号 English \\
  \verb|\wuhao  | & \wuhao  五号字 English \\
  \verb|\xiaowu | & \xiaowu 小五号 English \\
  \verb|\liuhao  | & \liuhao  六号字 English \\
  \verb|\xiaoliu | & \xiaoliu 小六号 English \\
  \verb|\qihao  | & \qihao  七号字 English \\
  \verb|\bahao | & \bahao 八号字 English \\

\end{tabular}

\subsection{调节字体}

需要说明的是由于Linux或者macOS系统上不包含学校写作指导要求的字体,因此我们将Windows的字体放在了fonts目录下,
由于这些字体具有版权,所以我们仍\textbf{强烈建议}您在Windows操作系统上编译最终版论文。

中文可选字体以及选用指令如下:

\begin{tabular}{l l}
  \verb|\songti| & {\songti 宋体} \\
  \verb|\heiti| & {\heiti 黑体}
\end{tabular}

我们在模板中通过调整\verb|\newCJKfontfamily|的AutoFakeBold参数来简单实现字体加粗,
在正文中你可以使用习惯的\verb|\textbf|指令来加粗对应中文。
如果你需要调整字体,也可以组合使用字体选择并加粗,比如\verb|\heiti\bfseries|

\textbf{宋体加粗测试},宋体不加粗测试。

{\heiti\bfseries 黑体加粗测试。}{\heiti 黑体不加粗测试。}

目前模板并没有按照一些其他模板写法中常见的,重定向加粗和倾斜效果到upright。


\section{模板主要结构}

本项目模板的主要结构, 如下表所示:
% TODO 进一步完善

\begin{table}[ht]
  \centering
  \begin{tabular}{r|l|l}
    \hline\hline
    \multicolumn{2}{l|}{main.tex } & 主文档,可以理解为文章入口。  \\ \hline
                                            & abstact.tex    & 中/英文摘要内容    \\ \cline{2-3}
                                            & acknowledgements.tex       & 致谢 \\ \cline{2-3}
                                            & appendix.tex       & 附录 \\ \cline{2-3}
                                            & conclusion.tex       & 结论 \\ \cline{2-3}
                                            & ref.bib       & 参考文献库 \\ \cline{2-3}
                                            & reference.tex       & 参考文献 \\ \cline{2-3}
    \raisebox{1em}{content 目录 }            & chapters 目录   & 章节内容           \\ \hline
    \multicolumn{2}{l|}{images 目录}         & 用于存放图片文件                                  \\ \hline
    \multicolumn{2}{l|}{suepthesis.cls }    & 模板入口                                          \\ \hline\hline
  \end{tabular}
\end{table}

我们不建议模板使用者更改原有模板的结构,
但如果您确实需要,请务必先充分阅读本模板的使用说明并了解相应的\LaTeX{}模板设计知识。
