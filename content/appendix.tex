%%
% The BIThesis Template for Bachelor Graduation Thesis
%
% 上海电力大学毕业设计(论文)中英文摘要 —— 使用 XeLaTeX 编译
%
% Copyright 2020-2023 SUEPaper
%
% This work may be distributed and/or modified under the
% conditions of the LaTeX Project Public License, either version 1.3
% of this license or (at your option) any later version.
% The latest version of this license is in
%   http://www.latex-project.org/lppl.txt
% and version 1.3 or later is part of all distributions of LaTeX
% version 2005/12/01 or later.
%
% This work has the LPPL maintenance status `maintained'.
%
% The Current Maintainer of this work is Haiwen Zhang.
%%

\begin{appendices}
    
    \textcolor{blue}{附录部分用于存放这里用来存放不适合放置在正文的大篇幅内容、典型如代码、图纸、完整数学证明过程等内容。}
  
    % 这里示范一下添加多个附录的方法:
    % 使用 \section 来添加一个附录
  
    \section{\LaTeX 环境的安装}
    \LaTeX 环境的安装。

    \newpage
    \section{堆溢出检测算法}
    \begin{algorithm}[h]
        \caption{堆溢出检测算法}\label{alg:ovf}
        \begin{algorithmic}[1]
            \IF {$\beta \in \mathbb{N^{*}} \land \Delta_\beta = \Delta_{\beta - 1} \land \beta < S$}
                \STATE 正常写入
            \ELSIF {$\beta \in \mathbb{N^{*}} \land \Delta_\beta \neq \Delta_{\beta - 1} \land \beta \geq S$}
                \STATE 发生堆溢出
            \ENDIF
        \end{algorithmic}
    \end{algorithm}
    
    \newpage
    \section{SUEPThesis 使用说明}
    SUEPThesis 使用说明。
  
    \textcolor{blue}{附录是毕业设计(论文)主体的补充项目,为了体现整篇文章的完整性,写入正文又可能有损于论文的条理性、逻辑性和精炼性,这些材料可以写入附录段,但对于每一篇文章并不是必须的。附录依次用大写正体英文字母 A、B、C……编序号,如附录 A、附录 B。阅后删除此段。}
  
    \textcolor{blue}{附录正文样式与文章正文相同:宋体、小四;行距:22 磅;间距段前段后均为 0 行。阅后删除此段。}
  
\end{appendices}