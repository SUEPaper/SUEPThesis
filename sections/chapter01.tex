\chapter{模板介绍}

Shanghai University of Electric Power \textbf{THESIS}
是根据《上海电力大学本科生毕业设计(论文)撰写规范》和《上海电力大学本科生毕业设计(论文)格式示范文本》(下文统一简称《规范》)编写的、
适用于上海电力大学学位论文写作的\emph{非官方} \LaTeX 模板。
目前版本(v\version{})提供了本科学位论文排版选项,且能够自动生成最终提交的打印版论文。

本文档将尽量详细地阐释  的使用方法和技巧。同时本文档直接使用  排版,其源代码文件 也可以作为一个实际样例以供读者参考使用。

目前已\emph{试验性地}加入对本科学位论文的支持,但仍亟需有上海电力大学本科论文排版经验的同学参与到 项目中。
我们也计划将该使用说明和模板文件 suepthesis.cls 使用 \textsf{DocStrip} 统一重构,并逐步向 \LaTeX3 迁移。
我们非常希望得到用户宝贵的反馈和建议,若您有意为 贡献 issues 和 pull requests,请移步至项目主页。

\section{文档排版样式说明}
本文档针对各部分不同内容使用不同的排版样式:文档正文使用宋体和英文衬线体(serif),
\emph{强调部分}使用\emph{楷体}和英文意大利体(\emph{italic}),
宏包名称使用英文无衬线体(\textsf{sans serif},例如 \textsf{hyperref}),
代码及选项使用英文等宽体(\texttt{typewriter})和\texttt{仿宋体}排版。

\subsection{测试}

早期,电力工作者缺乏理论的支持,基本上凭借一定的工作经验的积累,根据变压器的外部检查情况来判断其内部和外部故障。但仅仅这样单靠人工经验来判断故障难免会有很大的失误和纰漏。
后来,随着油中气体分析法的发展,变压器故障诊断技术得到了很大的提高,进入实验测试诊断阶段。电力工作者首先用气体色谱分析测得变压器油中溶解的各气体含量,然后根据各种判断方法对产生这些气体的故障原因作出解释。常用的判断方法有特征气体法和比值法。比值法中尤以罗杰斯法和三比值法最为常用。特征气体判断法反应了故障点发热使绝缘材料分解时的事物本质。故障点产生气体的特征是随着故障点的故障类型以及故障能量级别以及所设计的绝缘材料的不同而不同的。特征气体法根据测得的某种气体的含量以基准值的比较来粗略判断变压器的故障。而比值法则根据实验测得的各种气体的比值组合来进一步比较具体地判断变压器地故障。
虽然对油中溶解气体分析可以有效地探测变压器潜伏性故障,但是,在电力设备地故障原因,故障现象和故障机理间同时存在随机性和模糊性地不确定现象。仅仅依靠实验测试地诊断方式难于满足现有地工程应用的要求。实际上随机性地产生是由于实验测试数据的分散性和本来故障地因果关系不确定性等造成的,它主要反应了客观上的不确定性,模糊性的产生是由于测试数据在主观判断边界上的亦此亦彼性,它主要反应了人为主观理解上的不确定性。许多现有的诊断规则都太绝对化,无法有效的解决不精确性,不完全性,和不确定性信息。同时,传统的比值法也存在编码范围不够完善的问题,很大程度上制约了变压器故障诊断技术的发展。
近期,随着人工智能的发展,其广泛地应用于电力系统各个领域。人工智能在变压器故障诊断方面地应用极大地促进了诊断技术的提高。人工智能的应用可以分为专家系统和人工神经网这两个方向。
